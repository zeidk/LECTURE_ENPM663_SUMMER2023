% !TeX program = xelatex
% Author: Mark H. Olson
% Website: https://mholson.com
% Github: https://github.com/mholson

%=+=+=+=+=+=+=+=+=+=+=+=+=+=+=+=+=+=+=+=+=+=+=+=+=+=+=+=+=+=+=+=+=+=+=+=+=+=+=+=

%=-=-=-=-=-=-=-=-=-=-=-=-=-=-=-=-=-=-=-=-=-=-=-=-=-=-=-=-=-=-=-=-=-=-=-=-=-=-=-=
% PREAMBLE :: sthlmNordLightDemo.tex 
%=-=-=-=-=-=-=-=-=-=-=-=-=-=-=-=-=-=-=-=-=-=-=-=-=-=-=-=-=-=-=-=-=-=-=-=-=-=-=-=
%
% > > >	The following beamer class options are available
%		aspectratio=169		Change aspect ratio to 16:9	
%		bibref				Include bibliography
%		sectionpages		Show section pages
%		codeminted			use minted pkg for code printing instead of listings
%							(requires additional setup & Python installed)
% 		font sizes 			{8, 9, 10, 11, 12, 14, 17, 20} 11 Default
%
% > > > The following sthlmnord package options are available
%		mode				= dark (default)
%							= light
%=-=-=-=-=-=-=-=-=-=-=-=-=-=-=-=-=-=-=-=-=-=-=-=-=-=-=-=-=-=-=-=-=-=-=-=-=-=-=-=

\documentclass[usenames,11,dvipsnames,svgnames,x11names,aspectratio=1610,bibref]{beamer}
% > > > Bibliography File 
\newcommand{\bibfilename}{mhoreferences.bib}
% > > > Choose Theme
\usetheme[mode=light]{sthlmnord}
%\usetheme{sthlmnord}

\usepackage[normalem]{ulem}                 % \uline
\usepackage{multicol}
\usepackage{tikz}
% \usepackage[]{roboto-mono}
% \usepackage{enumitem}
\usepackage{xspace}
\usepackage{awesomebox}
% \usepackage[font=scriptsize,labelfont=bf]{caption}
\usepackage{subcaption} % for captionof
% \usepackage{cleveref}

% \usepackage{titlesec} % paragraph symbol \S
\usepackage{underscore}                     % to write underscores without escaping them
\usepackage[minted,listings]{tcolorbox}     % listing
\usepackage{minted}

\usemintedstyle{tango}
\usepackage{xparse}

\usepackage{paralist}                       % compactitem, compactenum


\usepackage{fontawesome5}

\newcommand{\cmark}{\ding{51}}%
\newcommand{\xmark}{\ding{55}}%


\newcommand{\myimportantbox}[1]{%
  \awesomebox[nordEleven]{2pt}{\faFire}{nordEleven}{#1}}
  
\newcommand{\mytipbox}[1]{%
  \awesomebox[nordEleven!50]{2pt}{\faLightbulb[regular]}{Tomato}{#1}}

%%%%%%%%%%%%%%%%%%%%%%%%%%%%%%%%%%%
% FONTS
%%%%%%%%%%%%%%%%%%%%%%%%%%%%%%%%%%%
\newfontfamily{\orbitron}{orbitron}[
    Path=./fonts/orbitron/,
    Extension = .otf,
    UprightFont = *-light,
    BoldFont = *-medium
    ]
    
\newfontfamily{\robotomono}{robotomono}[
    Path=./fonts/robotomono/,
    Extension = .ttf,
    UprightFont = *-Regular,
    BoldFont = *-Bold,
    BoldItalicFont = *-BoldItalic,
    ItalicFont = *-Italic,
    NFSSFamily=RobotoMonoFamily
    ]

% \newfontfamily\monaco{Arial}[NFSSFamily=ArialFamily]
% \setminted[matlab]{fontfamily=ArialFamily}

% \newfontfamily{\raleway}{Raleway}[
%     Path=./fonts/raleway/,
%     Extension = .ttf,
%     UprightFont = *-Medium,
%     BoldFont = *-Bold,
%     BoldItalicFont = *-BoldItalic,
%     ItalicFont = *-Italic
%     ]


    
    
    
% \newfontfamily{\ralewayII}{Raleway}[
%     Path=./fonts/raleway/,
%     Extension = .ttf,
%     UprightFont = *-Light,
%     BoldFont = *-SemiBold,
%     BoldItalicFont = *-SemiBoldItalic,
%     ItalicFont = *-LightItalic
%     ]

% \newfontfamily{\ralewayIII}{Raleway}[
%     Path=./fonts/raleway/,
%     Extension = .ttf,
%     UprightFont = *-Thin,
%     BoldFont = *-Bold,
%     BoldItalicFont = *-BoldItalic,
%     ItalicFont = *-Italic
%     ]
    


%%%%%%%%%%%%%%%%%%%%%%%%%%%%%%%%%%%
% code error
%%%%%%%%%%%%%%%%%%%%%%%%%%%%%%%%%%%
\definecolor{coderror}{HTML}{f06b6b}
\newcommand{\codeerror}{\textcolor{coderror}{\bf error}}
\newcommand{\codeerrorII}[1]{\textcolor{coderror}{\bf #1}}
%%%%%%%%%%%%%%%%%%%%%%%%%%%%%%%%%%%
% VINIK24 colors
%%%%%%%%%%%%%%%%%%%%%%%%%%%%%%%%%%%


% 000000
% 6f6776
% 9a9a97
% c5ccb8
% 8b5580
% c38890
% a593a5
% 666092
% 9a4f50
% c28d75
% 7ca1c0
% 416aa3
% 8d6268
% be955c
% 68aca9
% 387080
% 6e6962
% 93a167
% 6eaa78
% 557064
% 9d9f7f
% 7e9e99
% 5d6872
% 433455

\definecolor{vinik1}{HTML}{000000}
\newcommand{\vinikI}[1]{\textcolor{vinik1}{#1}}
\definecolor{vinik2}{HTML}{6f6776}
\newcommand{\vinikII}[1]{\textcolor{vinik2}{#1}}
\definecolor{vinik3}{HTML}{9a9a97}
\newcommand{\vinikIII}[1]{\textcolor{vinik3}{#1}}
\definecolor{vinik4}{HTML}{c5ccb8}
\newcommand{\vinikIV}[1]{\textcolor{vinik4}{#1}}
\definecolor{vinik5}{HTML}{8b5580}
\newcommand{\vinikV}[1]{\textcolor{vinik5}{#1}}
\definecolor{vinik6}{HTML}{c38890}
\newcommand{\vinikVI}[1]{\textcolor{vinik6}{#1}}
\definecolor{vinik7}{HTML}{a593a5}
\newcommand{\vinikVII}[1]{\textcolor{vinik7}{#1}}
\definecolor{vinik8}{HTML}{666092}
\newcommand{\vinikVIII}[1]{\textcolor{vinik8}{#1}}
\definecolor{vinik9}{HTML}{9a4f50}
\newcommand{\vinikIX}[1]{\textcolor{vinik9}{#1}}
\definecolor{vinik10}{HTML}{c28d75}
\newcommand{\vinikX}[1]{\textcolor{vinik10}{#1}}
\definecolor{vinik11}{HTML}{7ca1c0}
\newcommand{\vinikXI}[1]{\textcolor{vinik11}{#1}}
\definecolor{vinik12}{HTML}{416aa3}
\newcommand{\vinikXII}[1]{\textcolor{vinik12}{#1}}
\definecolor{vinik13}{HTML}{8d6268}
\newcommand{\vinikXIII}[1]{\textcolor{vinik13}{#1}}
\definecolor{vinik14}{HTML}{be955c}
\newcommand{\vinikXIV}[1]{\textcolor{vinik14}{#1}}
\definecolor{vinik15}{HTML}{68aca9}
\newcommand{\vinikXV}[1]{\textcolor{vinik15}{#1}}
\definecolor{vinik16}{HTML}{387080}
\newcommand{\vinikXVI}[1]{\textcolor{vinik16}{#1}}
\definecolor{vinik17}{HTML}{6e6962}
\newcommand{\vinikXVII}[1]{\textcolor{vinik17}{#1}}
\definecolor{vinik18}{HTML}{93a167}
\newcommand{\vinikXVIII}[1]{\textcolor{vinik18}{#1}}
\definecolor{vinik19}{HTML}{6eaa78}
\newcommand{\vinikXIX}[1]{\textcolor{vinik19}{#1}}
\definecolor{vinik20}{HTML}{557064}
\newcommand{\vinikXX}[1]{\textcolor{vinik20}{#1}}
\definecolor{vinik21}{HTML}{9d9f7f}
\newcommand{\vinikXXI}[1]{\textcolor{vinik21}{#1}}
\definecolor{vinik22}{HTML}{7e9e99}
\newcommand{\vinikXXII}[1]{\textcolor{vinik22}{#1}}
\definecolor{vinik23}{HTML}{5d6872}
\newcommand{\vinikXXIII}[1]{\textcolor{vinik23}{#1}}
\definecolor{vinik24}{HTML}{433455}
\newcommand{\vinikXXIV}[1]{\textcolor{vinik24}{#1}}

\newcommand{\syntaxgray}[1]{\textcolor{vinik23}{#1}}


\definecolor{definitioncolor}{HTML}{3C415C}
\newcommand{\mydefinitioncolor}[1]{\textcolor{definitioncolor}{#1}}



\definecolor{sectionchevron}{HTML}{76BA99}
\newcommand{\sectionColor}[1]{\textcolor{sectionchevron}{#1}}

\definecolor{subsectionchevron}{HTML}{ADCF9F}
\newcommand{\subsectionColor}[1]{\textcolor{subsectionchevron}{#1}}

\definecolor{subsubsectionchevron}{HTML}{CED89E}
\newcommand{\subsubsectionColor}[1]{\textcolor{subsubsectionchevron}{#1}}

\definecolor{customcolor}{HTML}{A77979}
\definecolor{citationcolor}{HTML}{F67280}
\newcommand{\citationcolor}[1]{\textcolor{nordEleven}{#1}}




%%%%%%%%%%%%%%%%%%%%%%%%%%%%%%%%%%%
% CARNIVAL32 colors
%%%%%%%%%%%%%%%%%%%%%%%%%%%%%%%%%%%


% 000000
% 6f6776
% 9a9a97
% c5ccb8
% 8b5580
% c38890
% a593a5
% 666092
% 9a4f50
% c28d75
% 7ca1c0
% 416aa3
% 8d6268
% be955c
% 68aca9
% 387080
% 6e6962
% 93a167
% 6eaa78
% 557064
% 9d9f7f
% 7e9e99
% 5d6872
% 433455

\definecolor{carnival19}{HTML}{4d528a}
\newcommand{\carnivalXIX}[1]{\textcolor{carnival19}{#1}}
\newcommand{\monster}{\textbf{\textcolor{DodgerBlue1}{rpg::Monster}}}
\newcommand{\vampire}{\textbf{\textcolor{BurntOrange}{rpg::Vampire}}}
\newcommand{\werewolf}{\textbf{\textcolor{WildStrawberry}{rpg::Werewolf}}}


\definecolor{carnival20}{HTML}{556a97}
\newcommand{\carnivalXX}[1]{\textcolor{carnival20}{#1}}




\definecolor{carnival21}{HTML}{5c81a3}
\newcommand{\carnivalXXI}[1]{\textcolor{carnival21}{#1}}
\definecolor{carnival22}{HTML}{7dadc8}
\newcommand{\carnivalXXII}[1]{\textcolor{carnival22}{#1}}
\definecolor{carnival23}{HTML}{b0d6d9}
\newcommand{\carnivalXXIII}[1]{\textcolor{carnival23}{#1}}
\definecolor{carnival24}{HTML}{ece6df}
\newcommand{\carnivalXXIV}[1]{\textcolor{carnival24}{#1}}
\definecolor{carnival25}{HTML}{cfccca}
\newcommand{\carnivalXXV}[1]{\textcolor{carnival25}{#1}}




\definecolor{iconColor}{HTML}{383E56}
\definecolor{iconWarningColor}{HTML}{FA370E}
\definecolor{iconOkColor}{HTML}{00A8CC}
\definecolor{iconBestPracticeColor}{HTML}{00A8CC}
\newcommand{\iconColor}[1]{\textcolor{iconColor}{#1}}
\newcommand{\iconWarningColor}[1]{\textcolor{iconWarningColor}{#1}}
\newcommand{\iconBestPracticeColor}[1]{\textcolor{DodgerBlue}{#1}}

\newcommand{\iconOKColor}[1]{\textcolor{iconOkColor}{#1}}

\definecolor{codecolor1}{HTML}{004822}
\newcommand{\codecolorI}[1]{\textcolor{DeepPink1}{\bf #1}}
\definecolor{codecolor2}{HTML}{FD3135}
\newcommand{\codecolorII}[1]{\textcolor{DeepSkyBlue4}{\bf #1}}
\definecolor{codecolor3}{HTML}{0E6B8C}
\newcommand{\codecolorIII}[1]{\textcolor{IndianRed1}{\bf #1}}


\newcommand{\crosmsgComment}[1]{{\color{nordThree}{\textit{#1}}}}
\definecolor{disclaimer}{HTML}{EBCB8B}  
\newcommand{\mydisclaimer}{{\color{disclaimer}{\footnotesize Lecture 3~}}}
%%%%%%%%%%%%%%%%%%%%%%%%%%%%%%%%%%%
% line number customized
\renewcommand{\theFancyVerbLine}{\textcolor{carnival19}{\sf\scriptsize\arabic{FancyVerbLine}}}
\newcommand{\myline}[1]{{\textcolor{carnival19}{#1}}}
%%%%%%%%%%%%%%%%%%%%%%%%%%%%%%%%%%%

%%%%%%%%%%%%%%%%%%%%%%%%%%%%%%%%%%%
% Directory tree
%%%%%%%%%%%%%%%%%%%%%%%%%%%%%%%%%%%
\usepackage[edges]{forest}

\newlength\Size
\setlength\Size{4pt}
\tikzset{%
  folder/.pic={%
  \node[inner sep=0pt] (folder) at (0.16,0)
    {\normalsize\textcolor{black}{\faFolderOpen}};
    % \filldraw [draw=folderborder, top color=folderbg!50, bottom color=folderbg] (-1.05*\Size,0.2\Size+5pt) rectangle ++(.75*\Size,-0.2\Size-5pt);
    % \filldraw [draw=folderborder, top color=folderbg!50, bottom color=folderbg] (-1.15*\Size,-\Size) rectangle (1.15*\Size,\Size);
  },
  file/.pic={%
  \node[inner sep=0pt] (file) at (0.13,0)
    {\normalsize\faFile*[regular]};
    % \filldraw [draw=folderborder, top color=folderbg!5, bottom color=folderbg!10] (-\Size,.4*\Size+5pt) coordinate (a) |- (\Size,-1.2*\Size) coordinate (b) -- ++(0,1.6*\Size) coordinate (c) -- ++(-5pt,5pt) coordinate (d) -- cycle (d) |- (c) ;
  },
}
\forestset{%
  declare autowrapped toks={pic me}{},
  declare boolean register={pic root},
  pic root=0,
  pic dir tree/.style={%
    for tree={%
      folder,
      font=\raleway,
      grow'=0,
      l=0,
    },
    before typesetting nodes={%
      for tree={%
        edge label+/.option={pic me},
        l=0,
      },
      if pic root={
        tikz+={
          \pic at ([xshift=\Size].west) {folder};
        },
        align={l}
      }{},
    },
  },
  pic me set/.code n args=2{%
    \forestset{%
      #1/.style={%
        inner xsep=10,
        pic me={pic {#2}},
      }
    }
  },
  pic me set={directory}{folder},
  pic me set={file}{file},
}



%%%%%%%%%%%%%%%%%%%%%%%%%%%%%%%%%%%
% Stuff for table of contents
%%%%%%%%%%%%%%%%%%%%%%%%%%%%%%%%%%%
\usetikzlibrary{positioning}
\usetikzlibrary{tikzmark}
\tikzset{
  section number/.style={
    draw=none,
    rectangle,    
    left color=nordZero,
    right color=nordZero,
    minimum size=1.5em,
    text=white,
  },
  section/.style={
    draw=none,
    rectangle,    
    % shading=section shading,
    minimum height=1.5em,
    minimum width=0.9\textwidth,
    text width=0.9\textwidth,
    text=black,
    align=left
  },
  subsection number/.style={
    draw=none,
    rectangle,    
    left color=nordTwo,
    right color=nordTwo,
    minimum size=0.5em,
    text=white,
  },
  subsection/.style={
    draw=none,
    rectangle,    
    % shading=subsection shading,
    minimum height=1.5em,
    minimum width=0.939\textwidth,
    text width=0.939\textwidth,
    text=black,
    align=left
  }
}

\setbeamertemplate{section in toc}{
    \tikz[baseline=-0.5ex]\node[section number]{\,\inserttocsectionnumber};%
  \,%
  \tikz[baseline=-0.5ex]\node[section]{\inserttocsection};
}
\setbeamertemplate{subsection in toc}{
\tikz[baseline=-0.5ex]\node[subsection number]{\,\inserttocsubsectionnumber};%
  \,%
  \tikz[baseline=-0.5ex]\node[subsection]{\inserttocsubsection};
}

%%%%%%%%%%%%%%%%%%%%%%%%%%%%%%%%%%%
% href stuff
%%%%%%%%%%%%%%%%%%%%%%%%%%%%%%%%%%%
\newcommand{\urllink}[2]{\href{#1}{\textcolor{DimGray}{\uline{#2}}}}

\newcommand{\myemail}[2]{\href{#1}{\cnordTwelve{{\scriptsize\faEnvelope[regular]~}\uline{#2}}}}
%%%%%%%%%%%%%%%%%%%%%%%%%%%%%%%%%%%
% listing
%%%%%%%%%%%%%%%%%%%%%%%%%%%%%%%%%%%
\tcbuselibrary{listings, minted, skins}
\tcbset{listing engine=minted}


%%%%%%%%%%%%%%%%%%%%%%%%%%%%%%%%%%%
% sec, secsec, secsecsec
%%%%%%%%%%%%%%%%%%%%%%%%%%%%%%%%%%%
\newcommand\myhrulefill{\cnordTen{\hrulefill}}
\newcommand\myhrulefillred{\cnordEleven{\hrulefill}}
% Configure style for custom doubled line
\newcommand*{\doublerule}{\hrule width \hsize height 1pt \kern 0.5mm \hrule width \hsize height 2pt}

% Configure function to fill line with doubled line
\newcommand\doublerulefill{\leavevmode\leaders\vbox{\hrule width .1pt\kern1pt\hrule}\hfill\kern0pt }

% \newcommand{\mydisclaimer}{\cnordTwelve{Disclaimer}}
\renewcommand\sec{{\cnordSix{\secname}\hfill\mydisclaimer} }

\newcommand\secsec{\cnordSix{\secname}~\sectionColor{\small\faChevronRight}~{\cnordFive{\small\subsecname}\hfill\mydisclaimer}}

\newcommand\secsecsec{\cnordSix{\secname}~\sectionColor{\small\faChevronRight}~{\cnordFive{\small\subsecname}}~\subsectionColor{\footnotesize\faChevronRight}~{\cnordFour{\footnotesize\subsubsecname}\hfill\mydisclaimer}}

\newcommand\secsecsecsec[1]{\cnordSix{\secname}~\sectionColor{\small\faChevronRight}~{\cnordFive{\small\subsecname}}~\subsectionColor{\footnotesize\faChevronRight}~{\cnordFour{\footnotesize\subsubsecname}}~\subsubsectionColor{\scriptsize\faChevronRight}~{\cnordFour{\scriptsize #1}\hfill\mydisclaimer}}


%%%%%%%%%%%%%%%%%%%%%%%%%%%%%%%%%%%%%%%%%%%%%%%%%%%%%%%
\newcommand{\myDefinitionIcon}{\iconColor{\faScroll}\xspace}
% \newcommandx{\myDefinitionHeader}[3][1=12pt, 2=12pt]{{\fontsize{#1}{#2}\selectfont \myDefinitionIcon #3} \doublerulefill}

% \newcommand{\myDefinitionHeader}[3][12pt]{\fontsize{#2}{#1}\selectfont #3}


% \newcommand\secsecsec{{\color{darkgray}{\normalsize{\secname}} \textcolor{Tomato3}{|}{{\color{darkgray!90}{\small\subsecname}}}\textcolor{Tomato3}{|}{\color{darkgray!70}{\footnotesize\subsubsecname}}\myhrulefill}}



    
%%%%%%%%%%%%%%%%%%%%%%%%%%%%%%%%%%%
% C++
%%%%%%%%%%%%%%%%%%%%%%%%%%%%%%%%%%%
\def\CC{{C\nolinebreak[4]\hspace{-.05em}\raisebox{.4ex}{\tiny\bf ++}}\xspace}

\def\CCNINETYEIGHT{{C\nolinebreak[4]\hspace{-.05em}\raisebox{.4ex}{\tiny\bf ++98}}\xspace}
\def\CNINETYNINE{{C\nolinebreak[4]\hspace{-.05em}\raisebox{.4ex}{\tiny\bf 99}}\xspace}
\def\CCZEROTHREE{{C\nolinebreak[4]\hspace{-.05em}\raisebox{.4ex}{\tiny\bf ++03}}\xspace}
\def\CCELEVEN{{C\nolinebreak[4]\hspace{-.05em}\raisebox{.4ex}{\tiny\bf ++11}}\xspace}
\def\CCFOURTEEN{{C\nolinebreak[4]\hspace{-.05em}\raisebox{.4ex}{\tiny\bf ++14}}\xspace}
\def\CCSEVENTEEN{{C\nolinebreak[4]\hspace{-.05em}\raisebox{.4ex}{\tiny\bf ++17}}\xspace}
\def\CCTWENTY{{C\nolinebreak[4]\hspace{-.05em}\raisebox{.4ex}{\tiny\bf ++20}}\xspace}
\def\SALARY{{Salary\nolinebreak[4]\hspace{-.05em}\raisebox{.4ex}{\tiny\bf ++}}\xspace}


%%%%%%%%%%%%%%%%%%%%%%%%%%%%%%%%%%%%%%%%%%%%%%
\newmintinline[mfoot]{cpp}{fontsize=\footnotesize, fontfamily=RobotoMonoFamily, escapeinside=||}
\newmintinline[pfoot]{python}{fontsize=\footnotesize, fontfamily=RobotoMonoFamily, escapeinside=||}
\newmintinline[mlarge]{cpp}{fontsize=\large,fontfamily=RobotoMonoFamily, escapeinside=||}
\newmintinline[mLarge]{cpp}{fontsize=\Large,fontfamily=RobotoMonoFamily, escapeinside=||}
\newmintinline[mhuge]{cpp}{fontsize=\huge,fontfamily=RobotoMonoFamily, escapeinside=||}
\newmintinline[mHuge]{cpp}{fontsize=\Huge,fontfamily=RobotoMonoFamily, escapeinside=||}
\newmintinline[msmall]{cpp}{fontsize=\small,fontfamily=RobotoMonoFamily, escapeinside=||}
\newmintinline[mnormal]{cpp}{fontsize=\fontsize{11}{12}\selectfont, fontfamily=RobotoMonoFamily,escapeinside=||}

\newmintinline[pnormal]{python}{fontsize=\fontsize{11}{12}\selectfont, fontfamily=RobotoMonoFamily, escapeinside=||}

\newmintinline[mscript]{cpp}{fontsize=\scriptsize,fontfamily=RobotoMonoFamily, escapeinside=||}

\newmintinline[pscript]{python}{fontsize=\scriptsize, fontfamily=RobotoMonoFamily, escapeinside=||}


\newmintinline[mtiny]{cpp}{fontsize=\tiny,fontfamily=RobotoMonoFamily, escapeinside=||}

%%%%%%%%%%%%%%%%%%%% BASH
\newcommand\bashNormal[1]{{\mintinline[fontsize=\normalsize]{bash}{#1}}}
\newcommand\bashSmall[1]{{\mintinline[fontsize=\small]{bash}{#1}}}
\newcommand\bashFootnote[1]{{\mintinline[fontsize=\fontsize{9}{11}\selectfont]{bash}{#1}}}
\newcommand\bashScript[1]{\mintinline[fontsize=\fontsize{7}{9}\selectfont]{bash}{#1}}




%%%%%%%%%%%%%%%%%%%%%%%%%%%%%%%%%%%%%%%%%%%%%%
\newtcblisting{cppscript}{listing only, 
    minted language=cpp, 
    % minted style=stata-light,
    minted style=tango,
    % minted bgcolor=\cnordSix,
    colback=white, enhanced, frame hidden,
    boxsep=0pt,top=-2.5mm,bottom=-2.5mm,left=-1mm,right=-1mm,
nobeforeafter,
    minted options={
    fontfamily=RobotoMonoFamily,
    bgcolor=nordSix,
    fontsize=\scriptsize, 
    tabsize=1, 
    frame=leftline,
    escapeinside=||,
    rulecolor=nordFourteen,
    framerule=2pt,
    breaklines, 
    autogobble}
}

%%%%%%%%%%%%%%%%%%%%%%%%%%%%%%%%%%%%%%%%%%%%%%
\newtcblisting{pyscript}{listing only, 
    minted language=python, 
    % minted style=stata-light,
    minted style=tango,
    % minted bgcolor=\cnordSix,
    colback=white, enhanced, frame hidden,
    boxsep=0pt,top=-2.5mm,bottom=-2.5mm,left=-1mm,right=-1mm,
nobeforeafter,
    minted options={
    fontfamily=RobotoMonoFamily,
    bgcolor=white,
    fontsize=\scriptsize, 
    tabsize=1, 
    frame=leftline,
    escapeinside=||,
    rulecolor=nordFourteen,
    framerule=2pt,
    breaklines, 
    autogobble}
}


\newtcblisting{bashscript}{listing only, 
    minted language=bash, 
    % minted style=stata-light,
    minted style=tango,
    % minted bgcolor=\cnordSix,
    colback=white, enhanced, frame hidden,
    boxsep=0pt,top=-2.5mm,bottom=-2.5mm,left=-1mm,right=-1mm,
    nobeforeafter,
    minted options={
    fontfamily=RobotoMonoFamily,
    bgcolor=white,
    fontsize=\scriptsize, 
    tabsize=1, 
    frame=leftline,
    escapeinside=||,
    rulecolor=nordFourteen,
    framerule=2pt,
    breaklines, 
    autogobble}
}


\newtcblisting{bashfoot}{listing only, 
    minted language=bash, 
    % minted style=stata-light,
    minted style=tango,
    % minted bgcolor=\cnordSix,
    colback=white, enhanced, frame hidden,
    boxsep=0pt,top=-2.5mm,bottom=-2.5mm,left=-1mm,right=-1mm,
nobeforeafter,
    minted options={
    fontfamily=RobotoMonoFamily,
    bgcolor=nordSix,
    fontsize=\footnotesize, 
    tabsize=1, 
    frame=leftline,
    escapeinside=||,
    rulecolor=nordFourteen,
    framerule=2pt,
    breaklines, 
    autogobble}
}

\newtcblisting{textfoot}{listing only, 
    minted language=text, 
    % minted style=stata-light,
    minted style=tango,
    % minted bgcolor=\cnordSix,
    colback=white, enhanced, frame hidden,
    boxsep=0pt,top=-2.5mm,bottom=-2.5mm,left=-1mm,right=-1mm,
nobeforeafter,
    minted options={
    fontfamily=RobotoMonoFamily,
    bgcolor=nordSix,
    fontsize=\footnotesize, 
    tabsize=1, 
    frame=leftline,
    escapeinside=||,
    rulecolor=nordFourteen,
    framerule=2pt,
    breaklines, 
    autogobble}
}

\newtcblisting{textscript}{listing only, 
    minted language=text, 
    minted style=stata-light,
    % minted style=tango,
    % minted bgcolor=\cnordSix,
    colback=white, enhanced, frame hidden,
    boxsep=0pt,top=-2.5mm,bottom=-2.5mm,left=-1mm,right=-1mm,
nobeforeafter,
    minted options={
    fontfamily=RobotoMonoFamily,
    bgcolor=white,
    fontsize=\fontsize{7}{9}\selectfont, 
    tabsize=1, 
    frame=leftline,
    escapeinside=||,
    rulecolor=DeepSkyBlue4,
    framerule=2pt,
    breaklines, 
    autogobble}
}


\newtcblisting{yamlscript}{listing only, 
    minted language=yaml, 
    minted style=stata-light,
    % minted style=tango,
    % minted bgcolor=\cnordSix,
    colback=white, enhanced, frame hidden,
    boxsep=0pt,top=-2.5mm,bottom=-2.5mm,left=-1mm,right=-1mm,
nobeforeafter,
    minted options={
    fontfamily=RobotoMonoFamily,
    bgcolor=white,
    fontsize=\fontsize{7}{9}\selectfont, 
    tabsize=1, 
    frame=leftline,
    escapeinside=||,
    rulecolor=Coral2,
    framerule=2pt,
    breaklines, 
    autogobble}
}

\newtcblisting{texttiny}{listing only, 
    minted language=text, 
    minted style=stata-light,
    % minted style=tango,
    % minted bgcolor=\cnordSix,
    colback=white, enhanced, frame hidden,
    boxsep=0pt,top=-2.5mm,bottom=-2.5mm,left=-1mm,right=-1mm,
nobeforeafter,
    minted options={
    fontfamily=RobotoMonoFamily,
    bgcolor=white,
    fontsize=\tiny, 
    tabsize=1, 
    frame=leftline,
    escapeinside=||,
    rulecolor=DeepSkyBlue4,
    framerule=2pt,
    breaklines, 
    autogobble}
}


\newtcblisting{cppscriptII}{listing only, 
    minted language=cpp, 
    % minted style=stata-light,
    minted style=tango,
    % minted bgcolor=\cnordSix,
    colback=white, enhanced, frame hidden,
    boxsep=0pt,top=-2.5mm,bottom=-2.5mm,left=-1mm,right=-1mm,
nobeforeafter,
    minted options={
    fontfamily=RobotoMonoFamily,
    bgcolor=nordSix,
    fontsize=\fontsize{7}{8}\selectfont, 
    tabsize=1, 
    frame=leftline,
    escapeinside=||,
    rulecolor=nordFourteen,
    framerule=2pt,
    breaklines, 
    autogobble}
}


\newtcblisting{cppscriptIILine}{listing only, 
    minted language=cpp, 
    % minted style=stata-light,
    minted style=tango,
    % minted bgcolor=\cnordSix,
    colback=white, enhanced, frame hidden,
    boxsep=0pt,top=-2.5mm,bottom=-2.5mm,left=-1mm,right=-1mm,
nobeforeafter,
    minted options={
    fontfamily=RobotoMonoFamily,
    bgcolor=nordSix,
    fontsize=\fontsize{7}{8}\selectfont, 
    tabsize=1, 
    frame=leftline,
    linenos,
    escapeinside=||,
    rulecolor=nordFourteen,
    framerule=2pt,
    breaklines, 
    autogobble}
}





\newtcblisting{cppscriptline}{listing only, 
    minted language=cpp, 
    minted style=tango,
    % minted bgcolor=\cnordSix,
    colback=white, enhanced, frame hidden,
    boxsep=0pt,top=-2.5mm,bottom=-2.5mm,left=-1mm,right=-1mm,
nobeforeafter,
    minted options={
    fontfamily=RobotoMonoFamily,
    bgcolor=nordSix,
    fontsize=\scriptsize, 
    tabsize=1, 
    frame=leftline,
    linenos,
    escapeinside=||,
    rulecolor=nordFourteen,
    framerule=2pt,
    breaklines, 
    autogobble}
}

\newtcblisting{cppscriptlineTwoFive}{listing only, 
    minted language=cpp, 
    minted style=tango,
    % minted bgcolor=\cnordSix,
    colback=white, enhanced, frame hidden,
    boxsep=0pt,top=-2.5mm,bottom=-2.5mm,left=-1mm,right=-1mm,
nobeforeafter,
    minted options={
    fontfamily=RobotoMonoFamily,
    bgcolor=nordSix,
    highlightlines={2, 5},
    highlightcolor=nordThirteen,
    fontsize=\scriptsize, 
    tabsize=1, 
    frame=leftline,
    escapeinside=||,
    rulecolor=nordFourteen,
    framerule=2pt,
    breaklines, 
    linenos,
    autogobble}
}

\newtcblisting{textline}{listing only, 
    minted language=text, 
    minted style=tango,
    % minted bgcolor=\cnordSix,
    colback=white, enhanced, frame hidden,
    boxsep=0pt,top=-2.5mm,bottom=-2.5mm,left=-1mm,right=-1mm,
nobeforeafter,
    minted options={
    bgcolor=nordSix,
    fontsize=\scriptsize, 
    tabsize=1, 
    frame=leftline,
    linenos,
    escapeinside=||,
    rulecolor=nordFourteen,
    framerule=2pt,
    breaklines, 
    autogobble}
}

% \newtcblisting{textscript}{listing only, 
%     minted language=text, 
%     minted style=tango,
%     % minted bgcolor=\cnordSix,
%     colback=white, enhanced, frame hidden,
%     boxsep=0pt,top=-2.5mm,bottom=-2.5mm,left=-1mm,right=-1mm,
% nobeforeafter,
%     minted options={
%     bgcolor=black,
%     fontsize=\tiny, 
%     escapeinside=||,
%     tabsize=1, 
%     breaklines, 
%     autogobble}
% }

\newtcblisting{cppnormal}{listing only, 
    minted language=cpp, 
    minted style=tango,
    % minted bgcolor=\cnordSix,
    colback=white, enhanced, frame hidden,
    boxsep=0pt,top=-2.5mm,bottom=-2.5mm,left=-1mm,right=-1mm,
    nobeforeafter,
    minted options={
    fontfamily=RobotoMonoFamily,
    bgcolor=nordSix,
    fontsize=\normalsize, 
    tabsize=1, 
    frame=leftline,
    escapeinside=||,
    rulecolor=nordFourteen,
    framerule=2pt,
    breaklines, 
    autogobble}
}



\lstdefinelanguage{ROSMSG}{
    keywordstyle=\color{blue},
    keywordstyle = [2]{\color{VioletRed4}},
    keywordstyle = [3]{\color{OrangeRed1}},
    alsoletter=0123456789,
    alsodigit = {_},
    keywords={uint8,string,float32,float64, bool, int8},
    otherkeywords = {KITTING, ASSEMBLY, {ASSEMBLY_FRONT}, ASSEMBLY_BACK, WAREHOUSE, COMBINED, ariac_msgs/KittingTask, ariac_msgs/AssemblyTask, ariac_msgs/CombinedTask, 0, 1, 2, 3},
    morekeywords = [2]{KITTING, ASSEMBLY, COMBINED, ASSEMBLY_FRONT, ASSEMBLY_BACK, WAREHOUSE, 0, 1, 2, 3},
    morekeywords = [3]{ariac_msgs/KittingTask, ariac_msgs/AssemblyTask, ariac_msgs/CombinedTask},
    sensitive=true, % keywords are not case-sensitive
} % 


\newtcblisting{cppnormalsyntax}{listing only, 
    minted language=cpp, 
    minted style=tango,
    % minted bgcolor=\cnordSix,
    colback=white, enhanced, frame hidden,
    boxsep=0pt,top=-2.5mm,bottom=-2.5mm,left=-1mm,right=-1mm,
nobeforeafter,
    minted options={
    fontfamily=RobotoMonoFamily,
    bgcolor=nordSix,
    fontsize=\normalsize, 
    tabsize=1, 
    frame=leftline,
    escapeinside=||,
    rulecolor=nordEleven,
    framerule=2pt,
    breaklines, 
    autogobble}
}

\newtcblisting{cppfootsyntax}{listing only, 
    minted language=cpp, 
    minted style=tango,
    % minted bgcolor=\cnordSix,
    colback=white, enhanced, frame hidden,
    boxsep=0pt,top=-2.5mm,bottom=-2.5mm,left=-1mm,right=-1mm,
nobeforeafter,
    minted options={
    fontfamily=RobotoMonoFamily,
    bgcolor=nordSix,
    fontsize=\footnotesize, 
    tabsize=1, 
    frame=leftline,
    escapeinside=||,
    rulecolor=nordEleven,
    framerule=2pt,
    breaklines, 
    autogobble}
}

\newtcblisting{cppscriptsyntax}{listing only, 
    minted language=cpp, 
    minted style=tango,
    % minted bgcolor=\cnordSix,
    colback=white, enhanced, frame hidden,
    boxsep=0pt,top=-2.5mm,bottom=-2.5mm,left=-1mm,right=-1mm,
nobeforeafter,
    minted options={
    fontfamily=RobotoMonoFamily,
    bgcolor=nordSix,
    fontsize=\scriptsize, 
    tabsize=1, 
    frame=leftline,
    escapeinside=||,
    rulecolor=nordEleven,
    framerule=2pt,
    breaklines, 
    autogobble}
}

\newtcblisting{cppsmallsyntax}{listing only, 
    minted language=cpp, 
    minted style=tango,
    % minted bgcolor=\cnordSix,
    colback=white, enhanced, frame hidden,
    boxsep=0pt,top=-2.5mm,bottom=-2.5mm,left=-1mm,right=-1mm,
nobeforeafter,
    minted options={
    fontfamily=RobotoMonoFamily,
    bgcolor=nordSix,
    fontsize=\small, 
    tabsize=1, 
    frame=leftline,
    escapeinside=||,
    rulecolor=nordEleven,
    framerule=2pt,
    breaklines, 
    autogobble}
}

\newtcblisting{cpplargesyntax}{listing only, 
    minted language=cpp, 
    minted style=tango,
    % minted bgcolor=\cnordSix,
    colback=white, enhanced, frame hidden,
    boxsep=0pt,top=-2.5mm,bottom=-2.5mm,left=-1mm,right=-1mm,
nobeforeafter,
    minted options={
    fontfamily=RobotoMonoFamily,
    bgcolor=nordSix,
    fontsize=\large, 
    tabsize=1, 
    frame=leftline,
    escapeinside=||,
    rulecolor=nordEleven,
    framerule=2pt,
    breaklines, 
    autogobble}
}

\newtcblisting{cpplarge}{listing only, 
    minted language=cpp, 
    minted style=tango,
    % minted bgcolor=\cnordSix,
    colback=white, enhanced, frame hidden,
    boxsep=0pt,top=-2.5mm,bottom=-2.5mm,left=-1mm,right=-1mm,
nobeforeafter,
    minted options={
    fontfamily=RobotoMonoFamily,
    bgcolor=nordSix,
    fontsize=\large, 
    tabsize=1, 
    frame=leftline,
    escapeinside=||,
    rulecolor=nordFourteen,
    framerule=2pt,
    breaklines, 
    autogobble}
}

\newtcblisting{cppfoot}{listing only, 
    minted language=cpp, 
    minted style=tango,
    % minted bgcolor=\cnordSix,
    colback=white, enhanced, frame hidden,
    boxsep=0pt,top=-2.5mm,bottom=-2.5mm,left=-1mm,right=-1mm,
nobeforeafter,
    minted options={
    fontfamily=RobotoMonoFamily,
    bgcolor=nordSix,
    fontsize=\footnotesize, 
    tabsize=1, 
    frame=leftline,
    escapeinside=||,
    rulecolor=nordFourteen,
    framerule=2pt,
    breaklines, 
    autogobble}
}


%%%%%%%%%%%%%%%%%%%%%%%%%%%%%%%%%%%%%%%%%%%%%%
\newtcblisting{pyscript}{listing only, 
    minted language=python, 
    % minted style=stata-light,
    minted style=tango,
    % minted bgcolor=\cnordSix,
    colback=white, enhanced, frame hidden,
    boxsep=0pt,top=-2.5mm,bottom=-2.5mm,left=-1mm,right=-1mm,
nobeforeafter,
    minted options={
    fontfamily=RobotoMonoFamily,
    bgcolor=nordSix,
    fontsize=\footnotesize, 
    tabsize=1, 
    frame=leftline,
    escapeinside=||,
    rulecolor=nordFourteen,
    framerule=2pt,
    breaklines, 
    autogobble}
}



\newtcblisting{cppfootgray}{listing only, 
    minted language=cpp, 
    minted style=tango,
    % minted bgcolor=\cnordSix,
    colback=white, enhanced, frame hidden,
    boxsep=0pt,top=-2.5mm,bottom=-2.5mm,left=-1mm,right=-1mm,
nobeforeafter,
    minted options={
    fontfamily=RobotoMonoFamily,
    bgcolor=nordSix,
    fontsize=\footnotesize, 
    tabsize=1, 
    frame=leftline,
    escapeinside=||,
    rulecolor=nordZero,
    framerule=2pt,
    breaklines, 
    autogobble}
}


\newtcblisting{cppfootline}{listing only, 
    minted language=cpp, 
    minted style=tango,
    % minted bgcolor=\cnordSix,
    colback=white, enhanced, frame hidden,
    boxsep=0pt,top=-2.5mm,bottom=-2.5mm,left=-1mm,right=-1mm,
nobeforeafter,
    minted options={
    fontfamily=RobotoMonoFamily,
    bgcolor=nordSix,
    fontsize=\footnotesize, 
    tabsize=1, 
    frame=leftline,
    escapeinside=||,
    rulecolor=nordFourteen,
    framerule=2pt,
    breaklines, 
    linenos,
    autogobble}
}

\newtcblisting{cppsmall}{listing only, 
    minted language=cpp, 
    minted style=tango,
    % minted bgcolor=\cnordSix,
    colback=white, enhanced, frame hidden,
    boxsep=0pt,top=-2.5mm,bottom=-2.5mm,left=-1mm,right=-1mm,
nobeforeafter,
    minted options={
    fontfamily=RobotoMonoFamily,
    bgcolor=nordSix,
    fontsize=\small, 
    tabsize=1, 
    frame=leftline,
    escapeinside=||,
    rulecolor=nordFourteen,
    framerule=2pt,
    breaklines, 
    autogobble}
}

\newtcblisting{pysmall}{listing only, 
    minted language=python, 
    minted style=tango,
    % minted bgcolor=\cnordSix,
    colback=white, enhanced, frame hidden,
    boxsep=0pt,top=-2.5mm,bottom=-2.5mm,left=-1mm,right=-1mm,
nobeforeafter,
    minted options={
    fontfamily=RobotoMonoFamily,
    bgcolor=nordSix,
    fontsize=\small, 
    tabsize=1, 
    frame=leftline,
    escapeinside=||,
    rulecolor=nordFourteen,
    framerule=2pt,
    breaklines, 
    autogobble}
}


\newtcblisting{cppsmallgray}{listing only, 
    minted language=cpp, 
    minted style=tango,
    % minted bgcolor=\cnordSix,
    colback=white, enhanced, frame hidden,
    boxsep=0pt,top=-2.5mm,bottom=-2.5mm,left=-1mm,right=-1mm,
nobeforeafter,
    minted options={
    fontfamily=RobotoMonoFamily,
    bgcolor=nordSix,
    fontsize=\small, 
    tabsize=1, 
    frame=leftline,
    escapeinside=||,
    rulecolor=nordZero,
    framerule=2pt,
    breaklines, 
    autogobble}
}

\newtcblisting{cpptiny}{listing only, 
    minted language=cpp, 
    minted style=nord,
    % minted bgcolor=\cnordSix,
    colback=white, enhanced, frame hidden,
    boxsep=0pt,top=-2.5mm,bottom=-2.5mm,left=-1mm,right=-1mm,
nobeforeafter,
    minted options={
    fontfamily=RobotoMonoFamily,
    bgcolor=nordSix,
    fontsize=\tiny, 
    tabsize=1, 
    frame=leftline,
    escapeinside=||,
    rulecolor=nordFourteen,
    framerule=2pt,
    breaklines, 
    autogobble}
}

\newtcblisting{bashscript}{
    listing only, 
    minted language=bash, 
    minted style=tango,
    colback=white, 
    enhanced, 
    frame hidden,
    boxsep=0pt, top=-2.5mm, bottom=-2.5mm, left=-1mm, right=-1mm,
    nobeforeafter,
    minted options={
    fontfamily=RobotoMonoFamily,
    bgcolor=white,
    fontsize=\scriptsize, 
    tabsize=1, 
    frame=leftline,
    escapeinside=||,
    rulecolor=nordZero,
    framerule=2pt,
    breaklines, 
    autogobble}
}


\newtcblisting{bashTinyListLine}{
    listing only, 
    minted language=bash, 
    minted style=tango,
    colback=white, 
    enhanced, 
    frame hidden,
    boxsep=0pt, top=-2.5mm, bottom=-2.5mm, left=-1mm, right=-1mm,
    nobeforeafter,
    minted options={
    fontfamily=RobotoMonoFamily,
    bgcolor=white,
    fontsize=\tiny, 
    tabsize=1, 
    frame=leftline,
    escapeinside=||,
    rulecolor=nordZero,
    highlightlines={2},
    highlightcolor = nordThirteen,
    framerule=2pt,
    breaklines, 
    autogobble}
}

\newtcblisting{bashTinyList}{
    listing only, 
    minted language=bash, 
    minted style=tango,
    colback=white, 
    enhanced, 
    frame hidden,
    boxsep=0pt, top=-2.5mm, bottom=-2.5mm, left=-1mm, right=-1mm,
    nobeforeafter,
    minted options={
    fontfamily=RobotoMonoFamily,
    bgcolor=white,
    fontsize=\tiny, 
    tabsize=1, 
    frame=leftline,
    escapeinside=||,
    rulecolor=nordZero,
    framerule=2pt,
    breaklines, 
    autogobble}
}



\newtcblisting{bashScriptListLine}{
    listing only, 
    minted language=bash, 
    minted style=tango,
    colback=white, 
    enhanced, 
    frame hidden,
    boxsep=0pt, top=-2.5mm, bottom=-2.5mm, left=-1mm, right=-1mm,
    nobeforeafter,
    minted options={
    fontfamily=RobotoMonoFamily,
    bgcolor=white,
    fontsize=\scriptsize, 
    tabsize=1, 
    frame=leftline,
    escapeinside=||,
    rulecolor=nordZero,
    highlightlines={2},
    highlightcolor = nordThirteen,
    framerule=2pt,
    breaklines, 
    autogobble}
}

\newtcblisting{bashScriptList}{
    listing only, 
    minted language=bash, 
    minted style=tango,
    colback=white, 
    enhanced, 
    frame hidden,
    boxsep=0pt, top=-2.5mm, bottom=-2.5mm, left=-1mm, right=-1mm,
    nobeforeafter,
    minted options={
    fontfamily=RobotoMonoFamily,
    bgcolor=white,
    fontsize=\scriptsize, 
    tabsize=1, 
    frame=leftline,
    escapeinside=||,
    rulecolor=nordZero,
    framerule=2pt,
    breaklines, 
    autogobble}
}



\newtcblisting{bashscriptsyntax}{
    listing only, 
    minted language=bash, 
    minted style=tango,
    colback=white, 
    enhanced, 
    frame hidden,
    boxsep=0pt, top=-2.5mm, bottom=-2.5mm, left=-1mm, right=-1mm,
    nobeforeafter,
    minted options={
    fontfamily=RobotoMonoFamily,
    bgcolor=white,
    fontsize=\scriptsize, 
    tabsize=1, 
    frame=leftline,
    escapeinside=||,
    rulecolor=nordEleven,
    framerule=2pt,
    breaklines, 
    autogobble}
}

\newtcblisting{cmakescript}{listing only, 
    minted language=cmake, 
    minted style=tango,
    % minted bgcolor=\cnordSix,
    colback=white, enhanced, frame hidden,
    boxsep=0pt,top=-2.5mm,bottom=-2.5mm,left=-1mm,right=-1mm,
nobeforeafter,
    minted options={
    fontfamily=RobotoMonoFamily,
    bgcolor=white,
    fontsize=\scriptsize, 
    tabsize=1, 
    frame=leftline,
    escapeinside=||,
    rulecolor=nordZero,
    framerule=2pt,
    breaklines, 
    autogobble}
}



%%%%%%%%%%%%%%%%%%%%%%%%%%%%%%%%%%%
% mynote, mybestpractice, etc
%%%%%%%%%%%%%%%%%%%%%%%%%%%%%%%%%%%

\renewcommand{\emph}[1]{\textcolor{vinik2}{\it #1}}
\newcommand\mycustom{DeepSkyBlue4}
\newcommand{\summaryuline}[1]{{\color{black}\uline{#1}}}
% \newcommand{\mynote}{{\summaryuline{\textsc{\textcolor{\mycustom}{Note}}}}}


\newcommand{\mybestpractice}{\iconColor{\faThumbsUp}\xspace}
\newcommand{\mytodo}{\textcolor{iconColor}{\faTasks}\xspace}
\newcommand{\mynote}{\iconColor{\faEdit}\xspace}
\newcommand{\mywarning}{\iconWarningColor{\faExclamationTriangle}\xspace}
\newcommand{\myinfo}{\iconColor{\faInfoCircle}\xspace}
\newcommand{\myquestion}{\textcolor{iconColor}{\faQuestionCircle}\xspace}
\newcommand{\myreminder}{\textcolor{iconColor}{\faBell}\xspace}
\newcommand{\mydefinition}{\textcolor{iconColor}{\faScroll}\xspace}
\newcommand{\mycodesyntax}{\textcolor{iconColor}{\faCode}\xspace}



\newcommand{\mybestpracticeW}{\textcolor{white}{\faThumbsUp}\xspace}
\newcommand{\mybadpractice}{\iconWarningColor{\faThumbsDown}\xspace}
\newcommand{\mytodoW}{\textcolor{white}{\faTasks}\xspace}
\newcommand{\mynoteW}{\textcolor{white}{\faEdit}\xspace}
\newcommand{\mywarningW}{\textcolor{white}{\faExclamationTriangle}\xspace}
\newcommand{\myinfoW}{\textcolor{white}{\faInfoCircle}\xspace}
\newcommand{\myquestionW}{\textcolor{white}{\faQuestionCircle}\xspace}
\newcommand{\myreminderW}{\textcolor{white}{\faBell}\xspace}
\newcommand{\mydefinitionW}{\textcolor{white}{\faScroll}\xspace}

% \newcommand{\mytodo}{{\summaryuline{\textsc{\textcolor{\mycustom}{Todo}}}}}

\newcommand{\myanswer}{\textcolor{iconColor}{\faCheckSquare[regular]}\xspace}

\newcommand{\mydef}[1]{\mydefinitioncolor{\it #1}}
% \textit{\mydefinitioncolor{

\newcommand{\myfunfact}{{\summaryuline{\textsc{\textcolor{\mycustom}{FunFact}}}}}
\newcommand{\myvsc}{{\summaryuline{\textsc{\textcolor{\mycustom}{VSC}}}}}
\newcommand{\myemph}[1]{\textcolor{customcolor}{\it #1}}
% \newcommand{\myempherror}[1]{{\textbf{\textcolor{Tomato1}{#1}}}}
\newcommand{\myemphcode}[1]{{\textbf{\textcolor{Tomato1}{#1}}}}

\newcommand{\myexample}{{\summaryuline{\textsc{\textcolor{\mycustom}{Example}}}}}
\newcommand{\myexercise}{{\summaryuline{\textsc{\textcolor{\mycustom}{Exercise}}}}}
\newcommand{\myguideline}{{\summaryuline{\textsc{\textcolor{\mycustom}{Guidelines}}}}}


\newtcbox{\myterminalNormal}{enhanced,nobeforeafter,tcbox raise base,boxrule=0.5pt,top=0mm,bottom=0mm,
  right=0mm,left=6mm,arc=1pt,boxsep=1pt,fontupper=\normalsize\raleway,before upper={\vphantom{dlg}},
  colframe=black,coltext=white,colback=black,
  overlay={\begin{tcbclipinterior}\fill[white] (frame.south west)
    rectangle node[text=BrickRed,font=\sffamily\bfseries\tiny,rotate=0] {\faTerminal} ([xshift=6mm]frame.north west);\end{tcbclipinterior}}}

\newtcbox{\myterminalSmall}{enhanced,nobeforeafter,tcbox raise base,boxrule=0.5pt,top=0mm,bottom=0mm,
  right=0mm,left=6mm,arc=1pt,boxsep=1pt,fontupper=\small\raleway,before upper={\vphantom{dlg}},
  colframe=black,coltext=white,colback=black,
  overlay={\begin{tcbclipinterior}\fill[white] (frame.south west)
    rectangle node[text=BrickRed,font=\sffamily\bfseries\tiny,rotate=0] {\faTerminal} ([xshift=6mm]frame.north west);\end{tcbclipinterior}}}
    
\newtcbox{\myterminalFoot}{enhanced,nobeforeafter,tcbox raise base,boxrule=0.5pt,top=0mm,bottom=0mm,
  right=0mm,left=6mm,arc=1pt,boxsep=1pt,fontupper=\footnotesize\raleway,before upper={\vphantom{dlg}},
  colframe=black,coltext=white,colback=black,
  overlay={\begin{tcbclipinterior}\fill[white] (frame.south west)
    rectangle node[text=BrickRed,font=\sffamily\bfseries\tiny,rotate=0] {\faTerminal} ([xshift=6mm]frame.north west);\end{tcbclipinterior}}}


    

    
\newtcbox{\myterminalScript}{enhanced,nobeforeafter,tcbox raise base,boxrule=0.5pt,top=0mm,bottom=0mm,
  right=0mm,left=6mm,arc=1pt,boxsep=1pt,fontupper=\scriptsize\raleway,before upper={\vphantom{dlg}},
  colframe=BrickRed,coltext=white,colback=black,
  overlay={\begin{tcbclipinterior}\fill[white] (frame.south west)
    rectangle node[text=BrickRed,font=\sffamily\bfseries\tiny,rotate=0] {\faTerminal} ([xshift=6mm]frame.north west);\end{tcbclipinterior}}}

\newtcbox{\mytoolNormal}{enhanced,nobeforeafter,tcbox raise base,boxrule=0.5pt,top=0mm,bottom=0mm,
  right=0mm,left=6mm,arc=1pt,boxsep=1pt,fontupper=\normalsize\raleway,before upper={\vphantom{dlg}},
  colframe=nordZero,coltext=nordZero,colback=white,
  overlay={\begin{tcbclipinterior}\fill[nordZero] (frame.south west)
    rectangle node[text=white,font=\sffamily\bfseries\scriptsize,rotate=0] {\faTools} ([xshift=6mm]frame.north west);\end{tcbclipinterior}}}

\newtcbox{\mytoolFoot}{enhanced,nobeforeafter,tcbox raise base,boxrule=0.5pt,top=0mm,bottom=0mm,
  right=0mm,left=6mm,arc=1pt,boxsep=1pt,fontupper=\footnotesize\raleway,before upper={\vphantom{dlg}},
  colframe=nordZero,coltext=nordZero,colback=white,
  overlay={\begin{tcbclipinterior}\fill[nordZero] (frame.south west)
    rectangle node[text=white,font=\sffamily\bfseries\scriptsize,rotate=0] {\faTools} ([xshift=6mm]frame.north west);\end{tcbclipinterior}}}

\newtcbox{\mytoolSmall}{enhanced,nobeforeafter,tcbox raise base,boxrule=0.5pt,top=0mm,bottom=0mm,
  right=0mm,left=6mm,arc=1pt,boxsep=1pt,fontupper=\small\raleway,before upper={\vphantom{dlg}},
  colframe=nordZero,coltext=nordZero,colback=white,
  overlay={\begin{tcbclipinterior}\fill[nordZero] (frame.south west)
    rectangle node[text=white,font=\sffamily\bfseries\scriptsize,rotate=0] {\faTools} ([xshift=6mm]frame.north west);\end{tcbclipinterior}}}

\newtcbox{\mytoolScript}{enhanced,nobeforeafter,tcbox raise base,boxrule=0.5pt,top=0mm,bottom=0mm,
  right=0mm,left=6mm,arc=1pt,boxsep=1pt,fontupper=\scriptsize\raleway,before upper={\vphantom{dlg}},
  colframe=nordZero,coltext=nordZero,colback=white,
  overlay={\begin{tcbclipinterior}\fill[nordZero] (frame.south west)
    rectangle node[text=white,font=\sffamily\bfseries\scriptsize,rotate=0] {\faTools} ([xshift=6mm]frame.north west);\end{tcbclipinterior}}}


\definecolor{roscolor}{HTML}{EBCB8B}
% \definecolor{roscolor}{HTML}{AEBDCA}

\newtcbox{\myNode}{enhanced,nobeforeafter,tcbox raise base,boxrule=0.5pt,top=0mm,bottom=0mm,
  right=0mm,left=6mm,arc=1pt,boxsep=1pt,before upper={\vphantom{dlg}},
  colframe=black,coltext=nordZero,colback=white,
  overlay={\begin{tcbclipinterior}\fill[roscolor] (frame.south west)
    rectangle node[text=black,font=\sffamily\bfseries\footnotesize,rotate=0] {n} ([xshift=6mm]frame.north west);\end{tcbclipinterior}}}


    
\newtcbox{\mynodeNormal}{enhanced,nobeforeafter,tcbox raise base,boxrule=0.5pt,top=0mm,bottom=0mm,
  right=0mm,left=6mm,arc=1pt,boxsep=1pt,fontupper=\normalsize\raleway,before upper={\vphantom{dlg}},
  colframe=black,coltext=nordZero,colback=white,
  overlay={\begin{tcbclipinterior}\fill[roscolor] (frame.south west)
    rectangle node[text=black,font=\sffamily\bfseries\scriptsize,rotate=0] {$\mathcal{N}$} ([xshift=6mm]frame.north west);\end{tcbclipinterior}}}

\newtcbox{\mynodeSmall}{enhanced,nobeforeafter,tcbox raise base,boxrule=0.5pt,top=0mm,bottom=0mm,
  right=0mm,left=6mm,arc=1pt,boxsep=1pt,fontupper=\small\raleway,before upper={\vphantom{dlg}},
  colframe=black,coltext=nordZero,colback=white,
  overlay={\begin{tcbclipinterior}\fill[roscolor] (frame.south west)
    rectangle node[text=black,font=\sffamily\bfseries\scriptsize,rotate=0] {$\mathcal{N}$} ([xshift=6mm]frame.north west);\end{tcbclipinterior}}}
    
\newtcbox{\mynodeFoot}{enhanced,nobeforeafter,tcbox raise base,boxrule=0.5pt,top=0mm,bottom=0mm,
  right=0mm,left=6mm,arc=1pt,boxsep=1pt,fontupper=\footnotesize\raleway,before upper={\vphantom{dlg}},
  colframe=black,coltext=nordZero,colback=white,
  overlay={\begin{tcbclipinterior}\fill[roscolor] (frame.south west)
    rectangle node[text=black,font=\sffamily\bfseries\scriptsize,rotate=0] {$\mathcal{N}$} ([xshift=6mm]frame.north west);\end{tcbclipinterior}}}

\newtcbox{\mynodeScript}{enhanced,nobeforeafter,tcbox raise base,boxrule=0.5pt,top=0mm,bottom=0mm,
  right=0mm,left=6mm,arc=1pt,boxsep=1pt,fontupper=\scriptsize\raleway,before upper={\vphantom{dlg}},
  colframe=black,coltext=nordZero,colback=white,
  overlay={\begin{tcbclipinterior}\fill[roscolor] (frame.south west)
    rectangle node[text=black,font=\sffamily\bfseries\footnotesize,rotate=0] {\raleway n} ([xshift=6mm]frame.north west);\end{tcbclipinterior}}}
    
    
\newtcbox{\myframeNormal}{enhanced,nobeforeafter,tcbox raise base,boxrule=0.5pt,top=0mm,bottom=0mm,
  right=0mm,left=6mm,arc=1pt,boxsep=1pt,fontupper=\normalsize\raleway,before upper={\vphantom{dlg}},
  colframe=black,coltext=nordZero,colback=white,
  overlay={\begin{tcbclipinterior}\fill[roscolor] (frame.south west)
    rectangle node[text=black,font=\sffamily\bfseries\scriptsize,rotate=0] {$\mathcal{F}$} ([xshift=6mm]frame.north west);\end{tcbclipinterior}}}

\newtcbox{\myframeSmall}{enhanced,nobeforeafter,tcbox raise base,boxrule=0.5pt,top=0mm,bottom=0mm,
  right=0mm,left=6mm,arc=1pt,boxsep=1pt,fontupper=\small\raleway,before upper={\vphantom{dlg}},
  colframe=black,coltext=nordZero,colback=white,
  overlay={\begin{tcbclipinterior}\fill[roscolor] (frame.south west)
    rectangle node[text=black,font=\sffamily\bfseries\scriptsize,rotate=0] {$\mathcal{F}$} ([xshift=6mm]frame.north west);\end{tcbclipinterior}}}
    
\newtcbox{\myframeFoot}{enhanced,nobeforeafter,tcbox raise base,boxrule=0.5pt,top=0mm,bottom=0mm,
  right=0mm,left=6mm,arc=1pt,boxsep=1pt,fontupper=\footnotesize\raleway,before upper={\vphantom{dlg}},
  colframe=black,coltext=nordZero,colback=white,
  overlay={\begin{tcbclipinterior}\fill[roscolor] (frame.south west)
    rectangle node[text=black,font=\sffamily\bfseries\scriptsize,rotate=0] {$\mathcal{F}$} ([xshift=6mm]frame.north west);\end{tcbclipinterior}}}

\newtcbox{\myframeScript}{enhanced,nobeforeafter,tcbox raise base,boxrule=0.5pt,top=0mm,bottom=0mm,
  right=0mm,left=6mm,arc=1pt,boxsep=1pt,fontupper=\scriptsize\raleway,before upper={\vphantom{dlg}},
  colframe=black,coltext=nordZero,colback=white,
  overlay={\begin{tcbclipinterior}\fill[roscolor] (frame.south west)
    rectangle node[text=black,font=\sffamily\bfseries\scriptsize,rotate=0] {$\mathcal{F}$} ([xshift=6mm]frame.north west);\end{tcbclipinterior}}}
    

%%%%%%%%%%%%%%%%%%%%%%%%%%%%%%%%%%%

\newtcbox{\myTopic}{enhanced,nobeforeafter,tcbox raise base,boxrule=0.5pt,top=0.2mm,bottom=0.2mm,
  right=0.2mm,left=6mm,arc=1pt,boxsep=1pt,before upper={\vphantom{dlg}},
  colframe=black,coltext=nordZero,colback=white,
  overlay={\begin{tcbclipinterior}\fill[roscolor] (frame.south west)
    rectangle node[text=black,font=\sffamily\bfseries\scriptsize,rotate=0] {\coco t} ([xshift=6mm]frame.north west);\end{tcbclipinterior}}}

    
\newtcbox{\mytopicNormal}{enhanced,nobeforeafter,tcbox raise base,boxrule=0.5pt,top=0mm,bottom=0mm,
  right=0mm,left=6mm,arc=1pt,boxsep=1pt,fontupper=\normalsize\raleway,before upper={\vphantom{dlg}},
  colframe=black,coltext=nordZero,colback=white,
  overlay={\begin{tcbclipinterior}\fill[roscolor] (frame.south west)
    rectangle node[text=black,font=\sffamily\bfseries\scriptsize,rotate=0] {$\mathcal{T}$} ([xshift=6mm]frame.north west);\end{tcbclipinterior}}}

\newtcbox{\mytopicSmall}{enhanced,nobeforeafter,tcbox raise base,boxrule=0.5pt,top=0mm,bottom=0mm,
  right=0mm,left=6mm,arc=1pt,boxsep=1pt,fontupper=\small\raleway,before upper={\vphantom{dlg}},
  colframe=black,coltext=nordZero,colback=white,
  overlay={\begin{tcbclipinterior}\fill[roscolor] (frame.south west)
    rectangle node[text=black,font=\sffamily\bfseries\scriptsize,rotate=0] {$\mathcal{T}$} ([xshift=6mm]frame.north west);\end{tcbclipinterior}}}
    
\newtcbox{\mytopicFoot}{enhanced,nobeforeafter,tcbox raise base,boxrule=0.5pt,top=0mm,bottom=0mm,
  right=0mm,left=6mm,arc=1pt,boxsep=1pt,fontupper=\footnotesize\raleway,before upper={\vphantom{dlg}},
  colframe=black,coltext=nordZero,colback=white,
  overlay={\begin{tcbclipinterior}\fill[roscolor] (frame.south west)
    rectangle node[text=black,font=\sffamily\bfseries\scriptsize,rotate=0] {$\mathcal{T}$} ([xshift=6mm]frame.north west);\end{tcbclipinterior}}}

\newtcbox{\mytopicScript}{enhanced,nobeforeafter,tcbox raise base,boxrule=0.5pt,top=0mm,bottom=0mm,
  right=0mm,left=6mm,arc=1pt,boxsep=1pt,fontupper=\scriptsize\raleway,before upper={\vphantom{dlg}},
  colframe=black,coltext=nordZero,colback=white,
  overlay={\begin{tcbclipinterior}\fill[roscolor] (frame.south west)
    rectangle node[text=black,font=\sffamily\bfseries\footnotesize,rotate=0] {\raleway t} ([xshift=6mm]frame.north west);\end{tcbclipinterior}}}

%%%%%%%%%%%%%%%%%%%%%%%%%%%%%%%%%%%
\newtcbox{\myparameterNormal}{enhanced,nobeforeafter,tcbox raise base,boxrule=0.5pt,top=0mm,bottom=0mm,
  right=0mm,left=6mm,arc=1pt,boxsep=1pt,fontupper=\normalsize\raleway,before upper={\vphantom{dlg}},
  colframe=black,coltext=nordZero,colback=white,
  overlay={\begin{tcbclipinterior}\fill[roscolor] (frame.south west)
    rectangle node[text=black,font=\sffamily\bfseries\scriptsize,rotate=0] {$\mathcal{P}$} ([xshift=6mm]frame.north west);\end{tcbclipinterior}}}

\newtcbox{\myparameterSmall}{enhanced,nobeforeafter,tcbox raise base,boxrule=0.5pt,top=0mm,bottom=0mm,
  right=0mm,left=6mm,arc=1pt,boxsep=1pt,fontupper=\small\raleway,before upper={\vphantom{dlg}},
  colframe=black,coltext=nordZero,colback=white,
  overlay={\begin{tcbclipinterior}\fill[roscolor] (frame.south west)
    rectangle node[text=black,font=\sffamily\bfseries\scriptsize,rotate=0] {$\mathcal{P}$} ([xshift=6mm]frame.north west);\end{tcbclipinterior}}}
    
\newtcbox{\myparameterFoot}{enhanced,nobeforeafter,tcbox raise base,boxrule=0.5pt,top=0mm,bottom=0mm,
  right=0mm,left=6mm,arc=1pt,boxsep=1pt,fontupper=\footnotesize\raleway,before upper={\vphantom{dlg}},
  colframe=black,coltext=nordZero,colback=white,
  overlay={\begin{tcbclipinterior}\fill[roscolor] (frame.south west)
    rectangle node[text=black,font=\sffamily\bfseries\scriptsize,rotate=0] {$\mathcal{P}$} ([xshift=6mm]frame.north west);\end{tcbclipinterior}}}

\newtcbox{\myparameterScript}{enhanced,nobeforeafter,tcbox raise base,boxrule=0.5pt,top=0mm,bottom=0mm,
  right=0mm,left=6mm,arc=1pt,boxsep=1pt,fontupper=\scriptsize\raleway,before upper={\vphantom{dlg}},
  colframe=black,coltext=nordZero,colback=white,
  overlay={\begin{tcbclipinterior}\fill[roscolor] (frame.south west)
    rectangle node[text=black,font=\sffamily\bfseries\scriptsize,rotate=0] {$\mathcal{P}$} ([xshift=6mm]frame.north west);\end{tcbclipinterior}}}
    
%%%%%%%%%%%%%%%%%%%%%%%%%%%%%%%%%%

\newtcbox{\myMessage}{enhanced,nobeforeafter,tcbox raise base,boxrule=0.5pt,top=0mm,bottom=0mm,
  right=0mm,left=6mm,arc=1pt,boxsep=1pt,before upper={\vphantom{dlg}},
  colframe=black,coltext=nordZero,colback=white,
  overlay={\begin{tcbclipinterior}\fill[roscolor] (frame.south west)
    rectangle node[text=black,font=\sffamily\bfseries\footnotesize,rotate=0] {m} ([xshift=6mm]frame.north west);\end{tcbclipinterior}}}

    
\newtcbox{\mymessageNormal}{enhanced,nobeforeafter,tcbox raise base,boxrule=0.5pt,top=0mm,bottom=0mm,
  right=0mm,left=6mm,arc=1pt,boxsep=1pt,fontupper=\normalsize\raleway,before upper={\vphantom{dlg}},
  colframe=black,coltext=nordZero,colback=white,
  overlay={\begin{tcbclipinterior}\fill[v] (frame.south west)
    rectangle node[text=black,font=\sffamily\bfseries\scriptsize,rotate=0] {$\mathcal{M}$} ([xshift=6mm]frame.north west);\end{tcbclipinterior}}}

\newtcbox{\mymessageSmall}{enhanced,nobeforeafter,tcbox raise base,boxrule=0.5pt,top=0mm,bottom=0mm,
  right=0mm,left=6mm,arc=1pt,boxsep=1pt,fontupper=\small\raleway,before upper={\vphantom{dlg}},
  colframe=black,coltext=nordZero,colback=white,
  overlay={\begin{tcbclipinterior}\fill[roscolor] (frame.south west)
    rectangle node[text=black,font=\sffamily\bfseries\scriptsize,rotate=0] {$\mathcal{M}$} ([xshift=6mm]frame.north west);\end{tcbclipinterior}}}
    
\newtcbox{\mymessageFoot}{enhanced,nobeforeafter,tcbox raise base,boxrule=0.5pt,top=0mm,bottom=0mm,
  right=0mm,left=6mm,arc=1pt,boxsep=1pt,fontupper=\footnotesize\raleway,before upper={\vphantom{dlg}},
  colframe=black,coltext=nordZero,colback=white,
  overlay={\begin{tcbclipinterior}\fill[roscolor] (frame.south west)
    rectangle node[text=black,font=\sffamily\bfseries\scriptsize,rotate=0] {$\mathcal{M}$} ([xshift=6mm]frame.north west);\end{tcbclipinterior}}}

\newtcbox{\mymessageScript}{enhanced,nobeforeafter,tcbox raise base,boxrule=0.5pt,top=0mm,bottom=0mm,
  right=0mm,left=6mm,arc=1pt,boxsep=1pt,fontupper=\scriptsize\raleway,before upper={\vphantom{dlg}},
  colframe=black,coltext=nordZero,colback=white,
  overlay={\begin{tcbclipinterior}\fill[roscolor] (frame.south west)
    rectangle node[text=black,font=\sffamily\bfseries\scriptsize,rotate=0] {$\mathcal{M}$} ([xshift=6mm]frame.north west);\end{tcbclipinterior}}}
%%%%%%%%%%%%%%%%%%%%%%%%%%%%%%%%%%

\newtcbox{\myKeyboardShortcut}{enhanced,nobeforeafter,tcbox raise base,boxrule=0.5pt,top=0mm,bottom=0mm,
  right=0mm,left=6mm,arc=1pt,boxsep=1pt,before upper={\vphantom{dlg}},
  colframe=OliveGreen,coltext=nordZero,colback=white,
  overlay={\begin{tcbclipinterior}\fill[OliveGreen] (frame.south west)
    rectangle node[text=white,font=\sffamily\bfseries\tiny,rotate=0] {\faKeyboard} ([xshift=6mm]frame.north west);\end{tcbclipinterior}}}


\newtcbox{\myROSPackage}{enhanced,nobeforeafter,tcbox raise base,boxrule=0.5pt,top=0mm,bottom=0mm,
  right=0mm,left=6mm,arc=1pt,boxsep=1pt, before upper={\vphantom{dlg}},
  colframe=nordZero,coltext=nordZero,colback=white,
  overlay={\begin{tcbclipinterior}\fill[nordZero] (frame.south west)
    rectangle node[text=white,font=\sffamily\bfseries\tiny,rotate=0] {\faToolbox} ([xshift=6mm]frame.north west);\end{tcbclipinterior}}}

\newtcbox{\myExe}{enhanced,nobeforeafter,tcbox raise base,boxrule=0.5pt,top=0mm,bottom=0mm,
  right=0mm,left=6mm,arc=1pt,boxsep=1pt,before upper={\vphantom{dlg}},
  colframe=IndianRed2,coltext=nordZero,colback=white,
  overlay={\begin{tcbclipinterior}\fill[IndianRed2] (frame.south west)
    rectangle node[text=white,font=\sffamily\bfseries\tiny,rotate=0] {\faPlay} ([xshift=6mm]frame.north west);\end{tcbclipinterior}}}
    
\newtcbox{\myTerminal}{enhanced,nobeforeafter,tcbox raise base,boxrule=0.5pt,top=0mm,bottom=0mm,
  right=0mm,left=6mm,arc=1pt,boxsep=1pt, fontupper=\sffamily, before upper={\vphantom{dlg}},
  colframe=DodgerBlue,coltext=white,colback=black,
  overlay={\begin{tcbclipinterior}\fill[white] (frame.south west)
    rectangle node[text=BrickRed,font=\sffamily\bfseries\tiny,rotate=0] {\faTerminal} ([xshift=6mm]frame.north west);\end{tcbclipinterior}}}

\newtcbox{\myTerminalBlank}{enhanced,nobeforeafter,tcbox raise base,boxrule=0.5pt,top=0mm,bottom=0mm,
  right=0mm,left=0mm,arc=1pt,boxsep=1pt, fontupper=\sffamily, before upper={\vphantom{dlg}},
  colframe=DodgerBlue,coltext=white,colback=black}
    
\newtcbox{\myFolder}{enhanced,nobeforeafter,tcbox raise base,boxrule=0.5pt,top=0mm,bottom=0mm,
  right=0mm,left=6mm,arc=1pt,boxsep=1pt,before upper={\vphantom{dlg}},
  colframe=nordZero,coltext=nordZero,colback=white,
  overlay={\begin{tcbclipinterior}\fill[nordZero] (frame.south west)
    rectangle node[text=white,font=\sffamily\bfseries\tiny,rotate=0] {\faFolder[regular]} ([xshift=6mm]frame.north west);\end{tcbclipinterior}}}

\newtcbox{\myFile}{enhanced,nobeforeafter,tcbox raise base,boxrule=0.5pt,top=0mm,bottom=0mm,
  right=0mm,left=6mm,arc=1pt,boxsep=1pt,before upper={\vphantom{dlg}},
  colframe=nordZero,coltext=nordZero,colback=white,
  overlay={\begin{tcbclipinterior}\fill[nordZero] (frame.south west)
    rectangle node[text=white,font=\sffamily\bfseries\tiny,rotate=0] {\faFile*[regular]} ([xshift=6mm]frame.north west);\end{tcbclipinterior}}}
    
\newtcbox{\myTool}{enhanced,nobeforeafter,tcbox raise base,boxrule=0.5pt,top=0mm,bottom=0mm,
  right=0mm,left=6mm,arc=1pt,boxsep=1pt, before upper={\vphantom{dlg}},
  colframe=nordZero,coltext=nordZero,colback=white,
  overlay={\begin{tcbclipinterior}\fill[nordZero] (frame.south west)
    rectangle node[text=white,font=\sffamily\bfseries\tiny,rotate=0] {\faTools} ([xshift=6mm]frame.north west);\end{tcbclipinterior}}}

\newtcbox{\mypackageNormal}{enhanced,nobeforeafter,tcbox raise base,boxrule=0.5pt,top=0mm,bottom=0mm,
  right=0mm,left=6mm,arc=1pt,boxsep=1pt,fontupper=\normalsize\raleway,before upper={\vphantom{dlg}},
  colframe=nordZero,coltext=nordZero,colback=white,
  overlay={\begin{tcbclipinterior}\fill[nordZero] (frame.south west)
    rectangle node[text=white,font=\sffamily\bfseries\scriptsize,rotate=0] {\faToolbox} ([xshift=6mm]frame.north west);\end{tcbclipinterior}}}
    
\newtcbox{\mypackageLarge}{enhanced,nobeforeafter,tcbox raise base,boxrule=0.5pt,top=0mm,bottom=0mm,
  right=0mm,left=6mm,arc=1pt,boxsep=1pt,fontupper=\large\raleway,before upper={\vphantom{dlg}},
  colframe=nordZero,coltext=nordZero,colback=white,
  overlay={\begin{tcbclipinterior}\fill[nordZero] (frame.south west)
    rectangle node[text=white,font=\sffamily\bfseries\scriptsize,rotate=0] {\faToolbox} ([xshift=6mm]frame.north west);\end{tcbclipinterior}}}

\newtcbox{\mypackageSmall}{enhanced,nobeforeafter,tcbox raise base,boxrule=0.5pt,top=0mm,bottom=0mm,
  right=0mm,left=6mm,arc=1pt,boxsep=1pt,fontupper=\small\raleway,before upper={\vphantom{dlg}},
  colframe=nordZero,coltext=nordZero,colback=white,
  overlay={\begin{tcbclipinterior}\fill[nordZero] (frame.south west)
    rectangle node[text=white,font=\sffamily\bfseries\scriptsize,rotate=0] {\faToolbox} ([xshift=6mm]frame.north west);\end{tcbclipinterior}}}

\newtcbox{\mypackageFoot}{enhanced,nobeforeafter,tcbox raise base,boxrule=0.5pt,top=0mm,bottom=0mm,
  right=0mm,left=6mm,arc=1pt,boxsep=1pt,fontupper=\footnotesize\raleway,before upper={\vphantom{dlg}},
  colframe=nordZero,coltext=nordZero,colback=white,
  overlay={\begin{tcbclipinterior}\fill[nordZero] (frame.south west)
    rectangle node[text=white,font=\sffamily\bfseries\scriptsize,rotate=0] {\faToolbox} ([xshift=6mm]frame.north west);\end{tcbclipinterior}}}
    
\newtcbox{\mypackageScript}{enhanced,nobeforeafter,tcbox raise base,boxrule=0.5pt,top=0mm,bottom=0mm,
  right=0mm,left=6mm,arc=1pt,boxsep=1pt,fontupper=\scriptsize\raleway,before upper={\vphantom{dlg}},
  colframe=nordZero,coltext=nordZero,colback=white,
  overlay={\begin{tcbclipinterior}\fill[nordZero] (frame.south west)
    rectangle node[text=white,font=\sffamily\bfseries\scriptsize,rotate=0] {\faToolbox} ([xshift=6mm]frame.north west);\end{tcbclipinterior}}}
    

\newtcbox{\mykbNormal}{enhanced,nobeforeafter,tcbox raise base,boxrule=0.5pt,top=0mm,bottom=0mm,
  right=0mm,left=6mm,arc=1pt,boxsep=1pt,fontupper=\normalsize\raleway,before upper={\vphantom{dlg}},
  colframe=nordZero,coltext=nordZero,colback=white,
  overlay={\begin{tcbclipinterior}\fill[nordZero] (frame.south west)
    rectangle node[text=white,font=\sffamily\bfseries\scriptsize,rotate=0] {\faKeyboard} ([xshift=6mm]frame.north west);\end{tcbclipinterior}}}

\newtcbox{\mykbSmall}{enhanced,nobeforeafter,tcbox raise base,boxrule=0.5pt,top=0mm,bottom=0mm,
  right=0mm,left=6mm,arc=1pt,boxsep=1pt,fontupper=\small\raleway,before upper={\vphantom{dlg}},
  colframe=nordZero,coltext=nordZero,colback=white,
  overlay={\begin{tcbclipinterior}\fill[nordZero] (frame.south west)
    rectangle node[text=white,font=\sffamily\bfseries\scriptsize,rotate=0] {\faKeyboard} ([xshift=6mm]frame.north west);\end{tcbclipinterior}}}

\newtcbox{\mykbFoot}{enhanced,nobeforeafter,tcbox raise base,boxrule=0.5pt,top=0mm,bottom=0mm,
  right=0mm,left=6mm,arc=1pt,boxsep=1pt,fontupper=\footnotesize\raleway,before upper={\vphantom{dlg}},
  colframe=nordZero,coltext=nordZero,colback=white,
  overlay={\begin{tcbclipinterior}\fill[nordZero] (frame.south west)
    rectangle node[text=white,font=\sffamily\bfseries\scriptsize,rotate=0] {\faKeyboard} ([xshift=6mm]frame.north west);\end{tcbclipinterior}}}
    
\newtcbox{\mykbScript}{enhanced,nobeforeafter,tcbox raise base,boxrule=0.5pt,top=0mm,bottom=0mm,
  right=0mm,left=6mm,arc=1pt,boxsep=1pt,fontupper=\scriptsize\raleway,before upper={\vphantom{dlg}},
  colframe=nordZero,coltext=nordZero,colback=white,
  overlay={\begin{tcbclipinterior}\fill[nordZero] (frame.south west)
    rectangle node[text=white,font=\sffamily\bfseries\scriptsize,rotate=0] {\faKeyboard} ([xshift=6mm]frame.north west);\end{tcbclipinterior}}}
    



\newtcbox{\mydirNormal}{enhanced,nobeforeafter,tcbox raise base,boxrule=0.5pt,top=0mm,bottom=0mm,
  right=0mm,left=6mm,arc=1pt,boxsep=1pt,fontupper=\normalsize\raleway,before upper={\vphantom{dlg}},
  colframe=nordZero,coltext=nordZero,colback=white,
  overlay={\begin{tcbclipinterior}\fill[nordZero] (frame.south west)
    rectangle node[text=white,font=\sffamily\bfseries\scriptsize,rotate=0] {\faFolderOpen} ([xshift=6mm]frame.north west);\end{tcbclipinterior}}}    

\newtcbox{\mydirSmall}{enhanced,nobeforeafter,tcbox raise base,boxrule=0.5pt,top=0mm,bottom=0mm,
  right=0mm,left=6mm,arc=1pt,boxsep=1pt,fontupper=\small\raleway,before upper={\vphantom{dlg}},
  colframe=nordZero,coltext=nordZero,colback=white,
  overlay={\begin{tcbclipinterior}\fill[nordZero] (frame.south west)
    rectangle node[text=white,font=\sffamily\bfseries\scriptsize,rotate=0] {\faFolderOpen} ([xshift=6mm]frame.north west);\end{tcbclipinterior}}}
    
\newtcbox{\mydirFoot}{enhanced,nobeforeafter,tcbox raise base,boxrule=0.5pt,top=0mm,bottom=0mm,
  right=0mm,left=6mm,arc=1pt,boxsep=1pt,fontupper=\footnotesize\raleway,before upper={\vphantom{dlg}},
  colframe=nordZero,coltext=nordZero,colback=white,
  overlay={\begin{tcbclipinterior}\fill[nordZero] (frame.south west)
    rectangle node[text=white,font=\sffamily\bfseries\scriptsize,rotate=0] {\faFolderOpen} ([xshift=6mm]frame.north west);\end{tcbclipinterior}}}

\newtcbox{\mydirScript}{enhanced,nobeforeafter,tcbox raise base,boxrule=0.5pt,top=0mm,bottom=0mm,
  right=0mm,left=6mm,arc=1pt,boxsep=1pt,fontupper=\scriptsize\raleway,before upper={\vphantom{dlg}},
  colframe=nordZero,coltext=nordZero,colback=white,
  overlay={\begin{tcbclipinterior}\fill[nordZero] (frame.south west)
    rectangle node[text=white,font=\sffamily\bfseries\scriptsize,rotate=0] {\faFolderOpen} ([xshift=6mm]frame.north west);\end{tcbclipinterior}}}
    
    

    
\newtcbox{\myfileNormal}{enhanced,nobeforeafter,tcbox raise base,boxrule=0.5pt,top=0mm,bottom=0mm,
  right=0mm,left=6mm,arc=1pt,boxsep=1pt,fontupper=\normalsize\raleway,before upper={\vphantom{dlg}},
  colframe=nordZero,coltext=nordZero,colback=white,
  overlay={\begin{tcbclipinterior}\fill[nordZero] (frame.south west)
    rectangle node[text=white,font=\sffamily\bfseries\scriptsize,rotate=0] {\faFile*[regular]} ([xshift=6mm]frame.north west);\end{tcbclipinterior}}}

\newtcbox{\myfileSmall}{enhanced,nobeforeafter,tcbox raise base,boxrule=0.5pt,top=0mm,bottom=0mm,
  right=0mm,left=6mm,arc=1pt,boxsep=1pt,fontupper=\small\raleway,before upper={\vphantom{dlg}},
  colframe=nordZero,coltext=nordZero,colback=white,
  overlay={\begin{tcbclipinterior}\fill[nordZero] (frame.south west)
    rectangle node[text=white,font=\sffamily\bfseries\scriptsize,rotate=0] {\faFile*[regular]} ([xshift=6mm]frame.north west);\end{tcbclipinterior}}}

\newtcbox{\myfileFoot}{enhanced,nobeforeafter,tcbox raise base,boxrule=0.5pt,top=0mm,bottom=0mm,
  right=0mm,left=6mm,arc=1pt,boxsep=1pt,fontupper=\footnotesize\raleway,before upper={\vphantom{dlg}},
  colframe=nordZero,coltext=nordZero,colback=white,
  overlay={\begin{tcbclipinterior}\fill[nordZero] (frame.south west)
    rectangle node[text=white,font=\sffamily\bfseries\scriptsize,rotate=0] {\faFile*[regular]} ([xshift=6mm]frame.north west);\end{tcbclipinterior}}} 

\newtcbox{\myfileScript}{enhanced,nobeforeafter,tcbox raise base,boxrule=0.5pt,top=0mm,bottom=0mm,
  right=0mm,left=6mm,arc=1pt,boxsep=1pt,fontupper=\scriptsize\raleway,before upper={\vphantom{dlg}},
  colframe=nordZero,coltext=nordZero,colback=white,
  overlay={\begin{tcbclipinterior}\fill[nordZero] (frame.south west)
    rectangle node[text=white,font=\sffamily\bfseries\scriptsize,rotate=0] {\faFile*[regular]} ([xshift=6mm]frame.north west);\end{tcbclipinterior}}} 
    
    
% \newcommand{\myfile}[1]{{\textcolor{nordOne}{{\textsf[#1]}}}}
% \newcommand{\mydir}[1]{{\orbitron{\textcolor{black}{\footnotesize #1}}}}

\newcommand{\myemphII}[1]{{\textbf{\cnordZero{#1\xspace}}}}

% \newcommand{\mywarning}{\tcbox[enhanced,box align=base ,nobeforeafter,colback=white,colupper=Firebrick1,colframe=Firebrick1, size=small]{\textcolor{Firebrick1}{\ttfamily\bfseries \faWarning\xspace}}}

% > > >	Image File Paths
% 		Here you can add one or more paths to where your images are being
%		stored.  This will allow you to include only the image file 
%		name when placing it into your document.
%\graphicspath{{path1},{path2},{path3}}
\graphicspath{{./assets/}, {./figures/}} 
% > > >	Optional use of using subfiles to make content more modular
\usepackage{subfiles}
\usepackage{twoopt}
\usepackage{xargs}

% > > > Document Information
\title[ENPM663~|~ROS -- Part 2]{\cnordZero{ENPM663} -- \cnordTwo{L3: ROS -- Part 2}}
% \subtitle{[Lecture1] Course Introduction}
\newcommand{\titleAuthor}{Lecturer}
\author{Z. Kootbally \& C. Schlenoff}
\newcommand{\titleInstitute}{School}
\institute{University of Maryland}
\newcommand{\titleMiscI}{Course}
\newcommand{\descMiscI}{[ENPM663] Building a Manufacturing
Robot Software System}
% \newcommand{\titleMiscII}{File}
% \newcommand{\descMiscII}{\currfilebase}
\date{\today}
\titlegraphic{umdlogo.png}

% > > > pdf customizations via hyperref (pkg installed by beamer)
\hypersetup{
%colorlinks=true,
% You might want to disable color links for you final draft 
% AND for colors to work properly where links are involved.
colorlinks=false,
linkcolor={citationcolor},
citecolor={citationcolor},
urlcolor={citationcolor}
}


% \usepackage{xpatch}
% \makeatletter
% % replace executable python with python3
% \xpatchcmd\minted@pygmentize
%   {python -c}
%   {python3 -c}
%   {}{\fail}
% \xpatchcmd\minted@autogobble
%   {python -c}
%   {python3 -c}
%   {}{\fail}
% \makeatother

\crefname{page}{page}{pages}
\Crefname{page}{Page}{Pages}
%=-=-=-=-=-=-=-=-=-=-=-=-=-=-=-=-=-=-=-=-=-=-=-=-=-=-=-=-=-=-=-=-=-=-=-=-=-=-=-=
%
%    DOCUMENT BEGINS HERE 
%
%=-=-=-=-=-=-=-=-=-=-=-=-=-=-=-=-=-=-=-=-=-=-=-=-=-=-=-=-=-=-=-=-=-=-=-=-=-=-=-=

\usepackage{media9}
\begin{document}

\titlepage%
% \subfile{0-slides/tex.slide.sthmlNordCover}

%=-=-=-=-=-=-=-=-=-=-=-=-=-=-=-=-=-=-=-=-=-=-=-=-=-=-=-=-=-=-=-=-=-=-=-=-=-=-=-=
%   TABLE OF CONTENTS START   -=-=-=-=-=-=-=-=-=-=-=-=-=-=-=-=-=-=-=-=-=-=-=-=-=
% \begin{frame}
% 	\frametitle{Table of contents}
% 	% > > > For longer presentations use \tableofcontents[hideallsubsections] option
% 	%		It is also possible to manually control the entries of the table of
% 	% 		contents by sections.
% 	%\tableofcontents[sections={1-10}]
% 	%\framebreak
% 	%\tableofcontents[sections={11-15}]
% 	\tableofcontents
% \end{frame}



\section*{Table of contents}
\begin{frame}[fragile]{\sec}
\vspace{10pt}
\begin{multicols}{2}
{\tiny
\begin{itemize}
\item[]\tableofcontents[]
\end{itemize}
}
\end{multicols}
\end{frame}

%%%%%%%%%%%%%%%%%%%%%%%%%%%%%%%%
\section{Conventions}
%%%%%%%%%%%%%%%%%%%%%%%%%%%%%%%%
\begin{frame}[fragile]{\sec}
\vspace*{\fill}
\begin{center} 


{\large \myDefinitionIcon Conventions} \doublerulefill

\begin{compactitem}
\footnotesize
    \item This is a \urllink{ros.org}{link}
    \item This is a \myFile{file.txt}
    \item This is a \myFolder{folder}
    \item This is a \myTerminal{command}
    \item This is a \myKeyboardShortcut{keyboard shortcut}
\end{compactitem}

\hrulefill

\begin{compactitem}
\footnotesize
\item[] \mybestpractice Best practice.
% \item[] \mybadpractice Bad practice.
\item[] \mycodesyntax Code syntax.
\item[] \mydefinition Definition.
\item[] \mynote Important note.
% \item[] \myquestion Important question.
\item[] \myreminder Reminder.
\item[] \mytodo Task to do.
\item[] \mywarning Warning.
\end{compactitem}
\end{center}
\vspace*{\fill}
\end{frame}


%#####################################
\section{Node}
%#####################################
\begin{frame}[fragile]{\sec}
\vspace*{\fill}

{\large \myDefinitionIcon Node} \doublerulefill

\begin{center}
\emph{\CC Nodes need to be compiled while Python Nodes do not.}
\begin{compactitem}
\footnotesize
    \item Python Nodes can communicate with Python and \CC Nodes.
    \item \CC Nodes can communicate with Python and \CC Nodes.
\end{compactitem}
\end{center}




{\footnotesize \mytodo Run \CC Nodes }\doublerulefill

\begin{compactitem}
\footnotesize
    \item \CC \myTerminal{ros2 run demo_nodes_cpp talker}, \CC \myTerminal{ros2 run demo_nodes_cpp listener}
\end{compactitem}

{\footnotesize \mytodo  Run Python Nodes }\doublerulefill

\begin{compactitem}
\footnotesize
    \item Python \myTerminal{ros2 run demo_nodes_py talker}, Python \myTerminal{ros2 run demo_nodes_py listener}
\end{compactitem}

{\footnotesize \mytodo Run \CC and Python Nodes }\doublerulefill

\begin{compactitem}
\footnotesize
    \item \CC \myTerminal{ros2 run demo_nodes_cpp talker}, Python \myTerminal{ros2 run demo_nodes_py listener}
    \item Python \myTerminal{ros2 run demo_nodes_py talker}, \CC \myTerminal{ros2 run demo_nodes_cpp listener}
\end{compactitem}
\vspace*{\fill}
\end{frame}



%#####################################
\subsection{Message}
%#####################################
\begin{frame}[fragile]{\secsec}
\vspace*{\fill}

{\large \myDefinitionIcon Message} \doublerulefill

\begin{center}
\emph{Nodes communicate with each other by publishing Messages to Topics. A Message is a simple data structure, comprising typed fields.}
\end{center}



{\footnotesize \mytodo Display Published Messages }\doublerulefill

\begin{compactitem}
\footnotesize
    \item Run the Node: \myTerminal{ros2 run demo_nodes_cpp talker}
    \item Display Messages: \myTerminal{ros2 topic echo /chatter}
\end{compactitem}

{\footnotesize \mytodo  Visualize Message Definition }\doublerulefill

\begin{compactitem}
\small
     \item Get Message type: \myTerminal{ros2 node info /talker} or \myTerminal{ros2 topic type /chatter}
    \begin{bashscript}
    example_interfaces/msg/String
    \end{bashscript}
    
    \item Get Message definition: \myTerminal{ros2 interface show example_interfaces/msg/String}
    \begin{bashscript}
    string data
    \end{bashscript}
\item This means that if you want to publish a Message of type \myMessage{example_interfaces/msg/String} you will need to provide a string value for the field \emph{data}. If you want to retrieved a Message of type \myMessage{example_interfaces/msg/String} you will need to retrieve the value of the field \emph{data}.
\end{compactitem}

\vspace*{\fill}
\end{frame}



%#####################################
\subsection{Python}
\subsubsection*{Minimal Node}
%#####################################
\begin{frame}[fragile]{\secsecsec}
\vspace*{\fill}

{\large \mytodo Write a Minimal Python Node } \doublerulefill


\begin{compactitem}
\small
    \item Write a Node which simply displays a string on the standard output.
    \begin{tabular}{ l l}
    \footnotesize
    {\footnotesize Package} & \myROSPackage{first_package_py} \\
    {\footnotesize File} & \myFile{first_node.py} \\
    {\footnotesize Node} & \myNode{minimal_python_node} \\
    {\footnotesize Executable} & \myExe{first_node_exe} \\
    \end{tabular} 
\begin{pyscript}
import rclpy # ROS 2 Python client
from rclpy.node import Node # Node class


def main(args=None):
    # 1.Initialize ROS communications for a given context
    rclpy.init(args=args)
    # 2. Instantiate a Node
    node = Node('minimal_python_node')
    node.get_logger().info('Help me Obi-Wan Kenobi, you are my only hope')
    # 3. Shutdown a previously initialized context
    rclpy.shutdown()
\end{pyscript}

\mydefinition A context is an internal object used to keep track of some information/state.
\end{compactitem}



\vspace*{\fill}
\end{frame}

%#####################################
\begin{frame}[fragile]{\secsecsec}
\vspace*{\fill}

{\large \mytodo Execute a Node } \doublerulefill


\begin{compactitem}
\item \mybadpractice You can execute the script directly with \myTerminal{python3 first_node.py} but this is not the correct way.
\item \mybestpractice Execute it using ROS CLI.
\begin{compactenum}
    \item Create an entry point for the script.
\begin{pyscript}
entry_points={
    'console_scripts': [
        'first_node_exe = first_package_py.first_node:main'
    ],
},
\end{pyscript}
\item Build the package: \myTerminal{colcon build -{}-packages-select first_package_py}
\begin{compactitem}
    \item \myExe{first_node_exe} is generated in the \myFolder{install} folder.
\end{compactitem}
\item Run the Node: \myTerminal{ros2 run first_package_py first_node_exe}
\begin{compactitem}
    \item \myExe{first_node_exe} from the \myFolder{install} folder is executed.
\end{compactitem}
\end{compactenum}

\end{compactitem}


{\large \mytodo Inspect a Node } \doublerulefill

\begin{compactitem}
    \item \myTerminal{ros2 node info /minimal_python_node}
\end{compactitem}
\vspace*{\fill}
\end{frame}


%#####################################
\begin{frame}[fragile]{\secsecsec}
\vspace*{\fill}

{\large \mynote } \doublerulefill


\begin{compactitem}
\item The argument \myTerminalBlank{-{}-symlink-install} allows you to run Python Nodes without re-compiling them. An symbolic link to the original file is used.

\end{compactitem}
\vspace*{\fill}
\end{frame}



\subsubsection*{Publisher}
%#####################################
\begin{frame}[fragile]{\secsecsec}
\vspace*{\fill}

{\footnotesize \mytodo Write a Publisher }\doublerulefill

\begin{compactitem}
\small
    \item Write a Publisher which publishes a string every second.  
    
    \begin{tabular}{l l}
    \footnotesize
    {\footnotesize Package} & \myROSPackage{first_package_py} \\
    {\footnotesize File} & \myFile{publisher_py.py} \\
    {\footnotesize Topic} & \myTopic{my_chatter} \\
    {\footnotesize Node} & \myNode{publisher_py} \\
    {\footnotesize Message} & \myMessage{example_interfaces/msg/String} \\
    \end{tabular} 

    
    \item Edit \myFile{setup.py}
\begin{pyscript}
entry_points={
    'console_scripts': [
        'first_node_py = first_package_py.first_node:main',
        'publisher_py = first_package_py.publisher_py:main',
    ],
},
\end{pyscript}
\item Build the package: \myTerminal{colcon build -{}-packages-select first_package_py}
\item Run the Node: \myTerminal{ros2 run first_package_py publisher_py}
\item Introspect with \myTool{rqt_graph}
\end{compactitem}


\vspace*{\fill}
\end{frame}


%#####################################
\begin{frame}[fragile]{\secsecsec}
\vspace*{\fill}

{\footnotesize \bf \mycodesyntax \raleway  Code Explanation}\doublerulefill

\begin{compactitem}
\begin{pyscript}
self._publisher = self.create_publisher(String, 'my_chatter', 10)
\end{pyscript}
\begin{compactitem}
\item \urllink{https://docs.ros2.org/latest/api/rclpy/api/node.html\#rclpy.node.Node.create_publisher}{\pscript{create_publisher()}} constructs a Publisher object. 
\begin{compactitem}
    \item \pscript{String} is the type of the Message.
    \item \pscript{'my_chatter'} is the Topic to which the Publisher will publish Messages.
    \item \pscript{10} is the queue size: 10 Messages will be buffered before they are processed.
\end{compactitem}
\end{compactitem}


\begin{pyscript}
self._timer = self.create_timer(2, self.timer_callback)
\end{pyscript}
\begin{compactitem}
    \item \urllink{https://docs.ros2.org/latest/api/rclpy/api/node.html\#rclpy.node.Node.create_timer}{\pscript{create_timer()}} calls the  function \pscript{timer_callback()} every 2 seconds.
\begin{pyscript}
def timer_callback(self):
    self._msg.data = 'Help me Obi-Wan Kenobi, you are my only hope.'
    self._publisher.publish(self._msg)
    self.get_logger().info('Publishing: "%s"' % self._msg.data)
\end{pyscript}
\item \pscript{self._msg.data}: A value for the field \pscript{data} for the Message type \pscript{String} is provided.
\item \pscript{self._publisher.publish(self._msg)} publishes the Message to the topic.
\end{compactitem}
\end{compactitem}
\vspace*{\fill}
\end{frame}


%#####################################
\begin{frame}[fragile]{\secsecsec}
\vspace*{\fill}
\begin{center} 

{\footnotesize \bf \mytodo \raleway  Edit package.xml}\doublerulefill

\begin{mdframed}
\raleway
\footnotesize

Since you imported \mypackageScript{example_interfaces} in your code, you need to add this dependency in \myfileScript{package.xml}

Any time you import a ROS package you also have to add it to \myfileScript{package.xml}
\end{mdframed}

\begin{xmlscript}
<depend>example_interfaces</depend>
\end{xmlscript}



\end{center}
\vspace*{\fill}
\end{frame}

%#####################################
\subsubsection*{Subscriber}
%#####################################
\begin{frame}[fragile]{\secsecsec}
\vspace*{\fill}

{\footnotesize \bf \mytodo \raleway  Write a Subscriber}\doublerulefill

\begin{compactitem}
\small
    \item Write a Subscriber.

       \begin{tabular}{l l}
    \footnotesize
    {\raleway \scriptsize Package} & \mypackageScript{first_package_py} \\
    {\raleway \scriptsize File} & \myfileScript{subscriber_py.py} \\
    {\raleway \scriptsize Topic} & \mytopicScript{chatter663} \\
    {\raleway \scriptsize Node} & \mynodeScript{subscriber_py} \\
    {\raleway \scriptsize Message} & \mymessageScript{example_interfaces/msg/String} \\
    \end{tabular} 
    
    \item Edit \myfileScript{setup.py}
\begin{pyscript}
entry_points={
    'console_scripts': [
        'first_node_py = first_package_py.first_node:main',
        'publisher_py = first_package_py.publisher_py:main',
        'subscriber_py = first_package_py.subscriber_py:main',
    ],
},
\end{pyscript}
\item Build the package: \myterminalScript{colcon build -{}-packages-select first_package_py}
\item Run the Publisher: \myterminalScript{ros2 run first_package_py publisher_py}
\item Run the Subscriber: \myterminalScript{ros2 run first_package_py subscriber_py}
\item Introspect with \mytoolScript{rqt_graph}
\end{compactitem}


\vspace*{\fill}
\end{frame}


%#####################################
\begin{frame}[fragile]{\secsecsec}
\vspace*{\fill}

{\footnotesize \bf \mycodesyntax \raleway  Code Explanation}\doublerulefill

\begin{compactitem}
\begin{pyscript}
self._subscriber = self.create_subscription(String, 'chatter663', 
self.subscriber_callback, 10)
\end{pyscript}
\begin{compactitem}
\item \urllink{https://docs.ros2.org/latest/api/rclpy/api/node.html#rclpy.node.Node.create_subscription}{\pscript{create_subscription()}} creates a new subscription to a Topic. 
\begin{compactitem}
    \item \pscript{String} is the type of the Message to retrieve.
    \item \pscript{'chatter663'} is the Topic to listen to.
    \item \pscript{subscriber_callback} is the function to call whenever a Message is published to \pscript{'chatter663'}.
    \item \pscript{10} is the queue size: 10 Messages will be buffered before they are  processed.
\end{compactitem}
\end{compactitem}
\begin{pyscript}
def subscriber_callback(self, msg):
        self.get_logger().info('Receiving: "%s"' % msg.data)
\end{pyscript}
\begin{compactitem}
    \item This function unpacks the Message and prints the value for the field \pscript{data} on the standard output.
\end{compactitem}
\end{compactitem}
\vspace*{\fill}
\end{frame}


%#####################################
\subsection{\CC}
\subsubsection*{Simple Node}
%#####################################
\begin{frame}[fragile]{\secsecsec}
\vspace*{\fill}

{\footnotesize \bf \mytodo \raleway  Write a \CC Node}\doublerulefill

\begin{compactitem}
\small
    \item Write a Node which simply displays a string on the standard output.

    \begin{tabular}{ l l}
    \footnotesize
    {\raleway \scriptsize Package} & \mypackageScript{first_package_cpp} \\
    {\raleway \scriptsize File} & \myfileScript{first_node_cpp.cpp} \\
    {\raleway \scriptsize Node} & \mynodeScript{first_node_cpp} \\
    {\raleway \scriptsize Executable} & \myfileScript{first_node_cpp} \\
    \end{tabular} 
    
    \item Edit \myfileScript{CMakeLists.txt} to generate the executable \myfileScript{first_node_cpp}
\item Build the package: \myterminalScript{colcon build -{}-packages-select first_package_cpp}
\item Run the Node: \myterminalScript{ros2 run first_package_cpp first_node_cpp}
\begin{compactitem}
    \item[] \mywarning The argument \LightTermScript{first_node_cpp} is the executable generated from compilation.
\end{compactitem}
\item Inspect the Node: \myterminalScript{ros2 node info /first_node_cpp}
\end{compactitem}
\vspace*{\fill}
\end{frame}


%#####################################
\begin{frame}[fragile]{\secsecsec}
\vspace*{\fill}

{\footnotesize \bf \mycodesyntax \raleway  Code Explanation}\doublerulefill

\begin{compactitem}
\begin{cppscript}
auto node = std::make_shared<rclcpp::Node>("first_node_cpp");
\end{cppscript}
\begin{compactitem}
\item \mscript{rclcpp::Node} object is initialized as a shared pointer. The name of the Node is passed to the constructor.
\end{compactitem}
\end{compactitem}
\vspace*{\fill}
\end{frame}


%#####################################
\subsubsection*{Build Script}
\begin{frame}[fragile]{\secsecsec}
\vspace*{\fill}
\begin{center} 
{\footnotesize \bf \mycodesyntax \raleway  CMakeLists.txt}\doublerulefill

\begin{mdframed}
\footnotesize
\raleway
\myfileScript{CMakeLists.txt} is used by \mytoolScript{ament\_cmake} to build \CC executables, shared libraries (e.g., Gazebo plugins), and to install files in the folder \mydirScript{install}
\end{mdframed}

 
% \begin{compactitem}
% \item \mytodo Take a look at the \urllink{https://docs.ros.org/en/foxy/How-To-Guides/Ament-CMake-Documentation.html}{documentation} for \mytoolSmall{ament_cmake}
\begin{compactitem}
\item \textcolor{black}{\texttt{cmake_minimum_required(VERSION 3.8)}} informs about the minimum version of \mytoolfoot{cmake} required to work with this file.
\item The argument to \textcolor{black}{\texttt{project()}} will be the package name and must be identical to the package name in the \myfileFoot{package.xml}. 
\item The project setup is done by \texttt{ament_package()} and this call must occur exactly once per package (more information \urllink{https://docs.ros.org/en/foxy/How-To-Guides/Ament-CMake-Documentation.html}{here}). 
\item \mynote We need to edit this file once we create a Node in this package.
\end{compactitem}

\end{center}
\vspace*{\fill}
\end{frame}


%#####################################
\begin{frame}[fragile]{\secsecsec}
\vspace*{\fill}
\begin{center} 
{\footnotesize \bf \mytodo \raleway  Edit CMakeLists.txt}\doublerulefill


\begin{cmakescript}
cmake_minimum_required(VERSION 3.8)
project(first_package_cpp)

if(CMAKE_COMPILER_IS_GNUCXX OR CMAKE_CXX_COMPILER_ID MATCHES "Clang")
  add_compile_options(-Wall -Wextra -Wpedantic)
endif()

# find dependencies
find_package(ament_cmake REQUIRED)
find_package(rclcpp REQUIRED)


add_executable(first_node_cpp src/first_node_cpp.cpp)
ament_target_dependencies(first_node_cpp rclcpp)

install(TARGETS
first_node_cpp
DESTINATION lib/${PROJECT_NAME}
)

install(DIRECTORY include DESTINATION share/${PROJECT_NAME})

ament_package()
\end{cmakescript}

\end{center}
\vspace*{\fill}
\end{frame}

\subsubsection*{Publisher}
%#####################################
\begin{frame}[fragile]{\secsecsec}
\vspace*{\fill}

{\footnotesize \bf \mytodo \raleway  Write a Publisher}\doublerulefill

\begin{compactitem}
\small
    \item Write a Publisher which publishes a string every second.  
    
    \begin{tabular}{l l}
    \footnotesize
    {\raleway \scriptsize Package} & \mypackageScript{first_package_cpp} \\
    {\raleway \scriptsize File} & \myfileScript{publisher_cpp.cpp/.hpp} \\
    {\raleway \scriptsize Topic} & \mytopicScript{chatter663} \\
    {\raleway \scriptsize Node} & \mynodeScript{publisher_cpp} \\
    {\raleway \scriptsize Message} & \mymessageScript{example_interfaces/msg/String} \\
    {\raleway \scriptsize Executable} & \myfileScript{publisher_cpp} \\
    \end{tabular} 

    
    \item Edit \myfileScript{CMakeLists.txt} to generate the executable \myfileScript{publisher_cpp}
\item Build the package: \myterminalScript{colcon build -{}-packages-select first_package_cpp}
\item Run the Node: \myterminalScript{ros2 run first_package_cpp publisher_cpp}
\item Introspect with \mytoolScript{rqt_graph}
\end{compactitem}
\vspace*{\fill}
\end{frame}


%#####################################
\begin{frame}[fragile]{\secsecsec}
\vspace*{\fill}

{\footnotesize \bf \mycodesyntax \raleway  Code Explanation}\doublerulefill

\begin{compactitem}
\begin{cppscript}
publisher_ = this->create_publisher<example_interfaces::msg::String>("chatter663", 10);
\end{cppscript}
\begin{compactitem}
\item Create a Publisher to publish \mymessageScript{example_interfaces/msg/String} to \mytopicScript{chatter663}
\end{compactitem}
\begin{cppscript}
timer_ = this->create_wall_timer(std::chrono::milliseconds((int)(1000.0)), 
        std::bind(&PublisherNode::timer_callback, this));
\end{cppscript}
\begin{compactitem}
\item Call the function \mscript{timer_callback} every 1 second (1000 ms).
\item \mscript{timer_callback} fills out the Message and publish it.
\end{compactitem}
\end{compactitem}
\vspace*{\fill}
\end{frame}

%#####################################
\subsubsection*{Callback}
%#####################################
\begin{frame}[fragile]{\secsecsec}
\vspace*{\fill}

{\footnotesize \bf \mydefinition \raleway  Callback}\doublerulefill

\begin{mdframed}
\raleway
\footnotesize
A callback function is a function \mfoot{f} passed as argument to a function \mfoot{g}: \mfoot{g(f)}
\end{mdframed}



\begin{compactitem}
\footnotesize
\item You plan to call some function \mscript{g}, and at runtime you need it to invoke another function \mscript{f}. 
\item However, you cannot simply hardcode the name of \mscript{f} within \mscript{g}. 
\item \mscript{f} may not be known definitively at compile time, or perhaps \mscript{g} belongs to a third-party API that you cannot change and recompile.
\end{compactitem}

{\footnotesize \bf \mydefinition \raleway  Pass a Callback to a Function}\doublerulefill

\begin{mdframed}
\raleway
\footnotesize
To pass a callback function \mscript{f} to a function \mscript{g}, we can pass the address of \mscript{f} to \mscript{g} (i.e., a pointer to \mscript{f}) or use a lambda expression.
\end{mdframed}

\vspace*{\fill}
\end{frame}


%#####################################
\begin{frame}[fragile]{\secsecsec}
\vspace*{\fill}

\begin{cppscriptII}
#include <iostream>
#include <functional> // needed for std::function

// callback function
int add(int x, int y) {
    return x + y;
}

// callback function
int multiply(int x, int y) {
    return x * y;
}

int caller(int x, int y, int (*func)(int, int)) {
    return func(x, y);
}

// with std::function
//int caller(int x, int y, std::function<int(int, int)> func) {
//     return func(x, y);
//}

int main() {
    std::cout << caller(20, 10, &add) << '\n';
    std::cout << caller(20, 10, &multiply) << '\n';
}
\end{cppscriptII}

\vspace*{\fill}
\end{frame}


%#####################################
\begin{frame}[fragile]{\secsecsec}
\vspace*{\fill}

{\footnotesize \bf \mydefinition \raleway  \mfoot{create_wall_timer()}}\doublerulefill

\begin{mdframed}
\raleway
\footnotesize
The method \mfoot{create_wall_timer()} takes 2 required arguments: 1) Time interval between triggers of the callback, and 2) an object returned by \mfoot{std::bind()}.

\begin{cppscript}
this->create_wall_timer(std::chrono::milliseconds((int)(1000.0)),
    std::bind(&PublisherNode::timer_callback, 
    this));
\end{cppscript}
\end{mdframed}





\begin{compactitem}
\item \mfoot{create_wall_timer()} invokes \mfoot{std::bind()} in a loop at the specified interval. If period is 1, \mfoot{std::bind()}  is invoked every second, if period is 2,  \mfoot{std::bind()}  is invoked every 2 seconds, etc.
\item A pointer to the callback method is not passed to \mfoot{create_wall_timer()} but it is passed to  \mfoot{std::bind()}. In other words, \mfoot{std::bind()} is our function \mfoot{caller()} from the previous slide. 
\end{compactitem}
\vspace*{\fill}
\end{frame}

%#####################################
\begin{frame}[fragile]{\secsecsec}
\vspace*{\fill}

{\footnotesize \bf \mydefinition \raleway  \mfoot{std::bind()}}\doublerulefill

\begin{mdframed}
\raleway
\footnotesize
\mfoot{std::bind(callback, args)} generates a forwarding call wrapper for the callback function. Calling this wrapper is equivalent to invoking the callback function with some of its arguments bound to \mfoot{args}.
\end{mdframed}

\begin{compactitem}
\item Each argument of \mfoot{args} may either be bound to a value or be a placeholder:
\begin{compactitem}
\item If bound to a value, calling the returned function object will always use that value as argument.
\item If a placeholder, calling the returned function object forwards an argument passed to the call (the one whose order number is specified by the placeholder).
\end{compactitem}
\end{compactitem}
\vspace*{\fill}
\end{frame}

%#####################################
\begin{frame}[fragile]{\secsecsec}
\vspace*{\fill}

\begin{cppscriptII}
#include <iostream>
#include <functional>

// callback function
int subtract(int x, int y) {
    return x - y;
}

// callback function
int return_one(){
    return 1;
}

int main() {
    auto f = std::bind(&return_one); //f is a wrapper
    std::cout << f() << '\n'; //1
    
    auto g = std::bind(&subtract, 10,  std::placeholders::_1); //g is a wrapper
    std::cout << g(3) << '\n'; //x is already set to 10, here we assign 3 to y

    auto h = std::bind(&subtract, std::placeholders::_1, 3); //h is a wrapper
    std::cout << h(10) << '\n'; //y is already set to 3, here we assign 10 to x
    
    auto i = std::bind(&subtract, std::placeholders::_1, std::placeholders::_2); //i is a wrapper
    std::cout << i(10, 3) << '\n'; //we set x to 10 and y to 3
}
\end{cppscriptII}

\vspace*{\fill}
\end{frame}

%#####################################
\begin{frame}[fragile]{\secsecsec}
\vspace*{\fill}

{\footnotesize \bf \mynote \raleway  Summary}\doublerulefill

\begin{mdframed}
\raleway
\footnotesize
\begin{cppscript}
this->create_wall_timer(std::chrono::milliseconds((int)(1000.0)),
    std::bind(&PublisherNode::timer_callback, 
    this));
\end{cppscript}
\end{mdframed}

\begin{compactenum}

\item \mfoot{std::bind(&PublisherNode::timer_callback, this)} generates a wrapper for the method \mfoot{PublisherNode::timer_callback()}. The difference with the code from the previous slide is that this wrapper is not  stored in a variable (e.g., \mfoot{f}, \mfoot{g}, \mfoot{h}, and \mfoot{i} from the previous slide) and we are not explicitly calling this wrapper.  
% The wrapper generated from \mfoot{std::bind(&Robot::timer_callback, this)} will be directly called by \mfoot{create_wall_timer}.
\item \mfoot{create_wall_timer()} calls the wrapper every second in a loop. Since \mfoot{PublisherNode::timer_callback()} does not take any parameter, we are not providing any argument in this case (similar to the function \mfoot{return_one()} from the previous slide).
\begin{compactitem}
\item \mynote The keyword \mfoot{this} is used so the binding applies to the current object of the class \mfoot{PublisherNode}.
\end{compactitem}
\end{compactenum}
\vspace*{\fill}
\end{frame}

%#####################################
\subsubsection*{Build Scripts}
\begin{frame}[fragile]{\secsecsec}
\vspace*{\fill}
\begin{center} 
{\footnotesize \bf \mytodo \raleway  Edit CMakeLists.txt}\doublerulefill


\begin{cmakescript}
find_package(example_interfaces REQUIRED) 

add_executable(first_node_cpp src/first_node_cpp.cpp)
ament_target_dependencies(first_node_cpp rclcpp)

add_executable(publisher_cpp src/publisher_cpp.cpp)
ament_target_dependencies(publisher_cpp rclcpp example_interfaces)

install(TARGETS
first_node_cpp 
publisher_cpp
DESTINATION lib/${PROJECT_NAME}
)

\end{cmakescript}

{\footnotesize \bf \mytodo \raleway  Edit package.xml}\doublerulefill

\begin{xmlscript}
<depend>example_interfaces</depend>
\end{xmlscript}


\end{center}
\vspace*{\fill}
\end{frame}

%#####################################
% \begin{frame}[fragile]{\secsecsec}
% \vspace*{\fill}

% {\footnotesize \bf \mycodesyntax \raleway  Code Explanation}\doublerulefill

% \begin{compactitem}
% \begin{cppscript}
% publisher_ = this->create_publisher<std_msgs::msg::String>("chatter663", 10);
% \end{cppscript}
% \begin{compactitem}
% \item Create a Publisher to publish \mymessageScript{std_msgs/msg/String} to \mytopicScript{chatter663}
% \end{compactitem}
% \begin{cppscript}
% timer_ = this->create_wall_timer(std::chrono::milliseconds((int)(1000.0)), 
%         std::bind(&PublisherNode::timer_callback, this));
% \end{cppscript}
% \begin{compactitem}
% \item Call the function \mscript{timer_callback} every 1 second (1000 ms).
% \item \mscript{timer_callback} fills out the Message and publish it.
% \end{compactitem}
% \end{compactitem}
% \vspace*{\fill}
% \end{frame}


\subsubsection*{Subscriber}
%#####################################
\begin{frame}[fragile]{\secsecsec}
\vspace*{\fill}

{\footnotesize \bf \mytodo \raleway  Write a Subscriber}\doublerulefill

\begin{compactitem}
\small
    \item Write a Subscriber.  
    
    \begin{tabular}{l l}
    \footnotesize
    {\raleway \scriptsize Package} & \mypackageScript{first_package_cpp} \\
    {\raleway \scriptsize File} & \myfileScript{subscriber_cpp.cpp/.hpp} \\
    {\raleway \scriptsize Topic} & \mytopicScript{chatter663} \\
    {\raleway \scriptsize Node} & \mynodeScript{subscriber_cpp} \\
    {\raleway \scriptsize Message} & \mymessageScript{example_interfaces/msg/String} \\
    {\raleway \scriptsize Executable} & \myfileScript{subscriber_cpp} \\
    \end{tabular} 

    
    \item Edit \myfileScript{CMakeLists.txt} to generate the executable \myfileScript{subscriber_cpp}
\item Build the package: \myterminalScript{colcon build -{}-packages-select first_package_cpp}
\item Run the Node: \myterminalScript{ros2 run first_package_cpp publisher_cpp}
\item Run the Node: \myterminalScript{ros2 run first_package_cpp subscriber_cpp}
\item Introspect with \mytoolScript{rqt_graph}
\end{compactitem}
\vspace*{\fill}
\end{frame}


%#####################################
\begin{frame}[fragile]{\secsecsec}
\vspace*{\fill}

{\footnotesize \bf \mycodesyntax \raleway  Code Explanation}\doublerulefill

\begin{compactitem}
\begin{cppscript}
subscriber_ = this->create_subscription<example_interfaces::msg::String>("chatter663", 
10, std::bind(&SubscriberNode::chatter_callback, this, 
std::placeholders::_1));
\end{cppscript}
\begin{compactitem}
\item \mscript{create_subscription()} creates a subscription to \mytopicScript{chatter663}
\item \mscript{SubscriberNode::chatter_callback()} will be called for each arriving Message on \mytopicScript{chatter663}
\end{compactitem}
\end{compactitem}
\vspace*{\fill}
\end{frame}

%#####################################
\begin{frame}[fragile]{\secsecsec}
\vspace*{\fill}
\begin{center} 
{\footnotesize \bf \mytodo \raleway  Edit CMakeLists.txt}\doublerulefill


\begin{cmakescript}
find_package(example_interfaces REQUIRED) 

add_executable(first_node_cpp src/first_node_cpp.cpp)
ament_target_dependencies(first_node_cpp rclcpp)

add_executable(publisher_cpp src/publisher_cpp.cpp)
ament_target_dependencies(publisher_cpp rclcpp example_interfaces)

add_executable(subscriber_cpp src/subscriber_cpp.cpp)
ament_target_dependencies(subscriber_cpp rclcpp example_interfaces)

install(TARGETS
first_node_cpp 
publisher_cpp
subscriber_cpp
DESTINATION lib/${PROJECT_NAME}
)

\end{cmakescript}

{\footnotesize \bf \mytodo \raleway  Edit package.xml}\doublerulefill

\begin{xmlscript}
<depend>example_interfaces</depend>
\end{xmlscript}


\end{center}
\vspace*{\fill}
\end{frame}


% %%%%%%%%%%%%%%%%%%
% %   --- FRAME
% %%%%%%%%%%%%%%%%%%
% \subsubsection*{Edit CMakeLists.txt}
% \begin{frame}[fragile]{\secsecsec}
% \vspace*{\fill}
% \begin{center} 
% {\large \mytodo Edit CMakeLists.txt} \doublerulefill

% \begin{compactitem}
% \item You need to edit \myfileSmall{CMakeLists.txt} to build and install the executable.

% \begin{cmakescript}
% add_executable(hello src/main.cpp)
% ament_target_dependencies(hello rclcpp)

% install(TARGETS
%   hello
%   DESTINATION lib/${PROJECT_NAME}
% )
% \end{cmakescript}

% \end{compactitem}

% \end{center}
% \vspace*{\fill}
% \end{frame}


% \subsubsection*{Spinning a Node}
% \begin{frame}[fragile]{\secsecsec}
% \vspace*{\fill}
% \begin{center} 
% {\large \mydefinition Spinning a Node} \doublerulefill


% \begin{compactitem}
% \item \mfoot{rclcpp::spin(node)} pauses the program to keep your Node alive. 
% \item Without this function, the Node will execute the instructions and shuts down by itself.
% \item Usually we would like to keep a Node running while the robot is still working.
% \item \mynote \mfoot{rclcpp::spin(node)} only takes a shared pointer as argument and this is why we used a shared pointer in the program we wrote.
% \end{compactitem}

% \end{center}
% \vspace*{\fill}
% \end{frame}


% \subsubsection*{OOP}
% \begin{frame}[fragile]{\secsecsec}
% \vspace*{\fill}
% \begin{center} 


% \begin{compactitem}
% \item The code we wrote is fine for a simple Node.
% \item Nodes usually need to publish Messages and/or receive Messages.
% \item Messages need to be stored as Message objects to be processed later.
% \item OOP is a good solution for such complexity.
% \begin{compactitem}
% \item See a curated list of  \urllink{https://github.com/fkromer/awesome-ros2}{ROS2 projects} with examples using OOP.
% \end{compactitem}
% \end{compactitem}

% {\large \mytodo} \doublerulefill

% \begin{compactitem}
% \item Rewrite the Node using OOP.
% \item If \myfileSmall{*.h} files are created, you need to edit  \myfileSmall{CMakeLists.txt} so they can be found and installed.
% \begin{cmakescript}
% include_directories(
%   include/first_package
% )

% install(DIRECTORY
%         include
%         DESTINATION share/${PROJECT_NAME})
% \end{cmakescript}
% \end{compactitem}

% \end{center}
% \vspace*{\fill}
% \end{frame}



% \begin{frame}[fragile]{\secsecsec}
% \vspace*{\fill}
% \begin{center} 
% {\large \mytodo} \doublerulefill



% \begin{compactitem}
% \item Zip \mypackageSmall{first_package} and then delete the package.
% \item Delete the folders \mydirSmall{log}, \mydirSmall{build}, and \mydirSmall{install}
% \item Clone the folder \mydirSmall{enpm809y_fall2022} in \mydirSmall{\textasciitilde/ros2_ws/src}
% \item \myterminalSmall{cd \textasciitilde/ros2_ws/src}
% \item \myterminalSmall{git clone https://github.com/zeidk/enpm809y_fall2022.git}
% \end{compactitem}

% \end{center}
% \vspace*{\fill}
% \end{frame}

%%%%%%%%%%%%%%%%%%
%   --- FRAME
%%%%%%%%%%%%%%%%%%
% \section{Function for Package}
% \begin{frame}[fragile]{\sec}
% \vspace*{\fill}
% \begin{center} 

% \begin{bashTinyList}
% code_path=/home/zeid/ariac2023_fork

% function ariac(){
%   if [ "$1" = "-b" ]
%   then
%     cd $code_path
%     colcon build
%     source install/setup.zsh
%   elif [ "$1" = "-c" ]
%   then
%     cd $code_path
%     rm -rf build install log
%     colcon build
%     source install/setup.zsh
%   elif [ "$1" = "-r" ]
%   then
%     ros2 launch ariac_gazebo ariac.launch.py
%   else
%     source /opt/ros/galactic/setup.zsh
%     cd $code_path
%     source install/setup.zsh 
%     export _colcon_cd_root=/opt/ros/galactic
%     source /usr/share/colcon_argcomplete/hook/colcon-argcomplete.zsh
%     source /usr/share/colcon_cd/function/colcon_cd.sh
%     eval "$(register-python-argcomplete3 ros2)"
%     eval "$(register-python-argcomplete3 colcon)"
%   fi
% }

% ariac
% \end{bashTinyList}

% \end{center}
% \vspace*{\fill}
% \end{frame}

%#####################################
\subsection{Python and \CC Nodes}
\begin{frame}[fragile]{\secsec}
\vspace*{\fill}

{\footnotesize \bf \mynote \raleway  \CC and Python Package}\doublerulefill
\begin{mdframed}
\raleway \scriptsize
In the previous lecture we created \mypackageScript{first_package} for both \CC and Python Nodes. 
\end{mdframed}

{\footnotesize \bf \mynote \raleway  \CC Nodes}\doublerulefill
\begin{mdframed}
\raleway \scriptsize
To create and use \CC Nodes in this package, reuse the approach described for the package \\\mypackageScript{first_package_cpp} 
\end{mdframed}

{\footnotesize \bf \mynote \raleway  Python Nodes}\doublerulefill
\begin{mdframed}
\raleway \footnotesize
\begin{compactitem}
    \scriptsize
    \item \myinfo Below is just a recommendation on how to structure the package for Python Nodes.
\item The folder \mydirScript{first_package/first_package} only contains Python modules that will be imported (but not called). 
\item The folder \mydirScript{first_package/nodes} only contains Python Nodes which are called (but not imported). Files in this folder import modules located in \mydirScript{first_package/first_package}.
\begin{compactitem}
    \scriptsize
    \item In the files located in \mydirScript{nodes}, you need to add \emphnorm{\#!/usr/bin/env python3} at the top of each file. The reason is that in \mypackageScript{first_package_py}, the Python scripts were called through \myfileScript{setup.py}, but now they will directly be called.
\end{compactitem}
\end{compactitem}

\end{mdframed}
\vspace*{\fill}
\end{frame}

%####################################
\begin{frame}[fragile]{\secsec}
\vspace*{\fill}
{\footnotesize \bf \mynote \raleway Package Setup for Python Nodes}\doublerulefill
\begin{mdframed}
\raleway \scriptsize
Since we created a package for \CC, we need to do some extra work to be able to run Python Nodes.
\end{mdframed}

    \begin{compactitem}
    \footnotesize
        \item In \myfileScript{package.xml}
        \begin{xmlscript}
<buildtool_depend>ament_cmake_python</buildtool_depend>
        \end{xmlscript}
        \item In \myfileScript{CMakeLists.txt}
        \begin{cmakescript}
# Install Python modules
ament_python_install_package(${PROJECT_NAME} 
  SCRIPTS_DESTINATION lib/${PROJECT_NAME})

# Install Python executables
install(PROGRAMS
  nodes/publisher_py.py
  nodes/subscriber_py.py
  DESTINATION lib/${PROJECT_NAME}
)
        \end{cmakescript}
\item We previously used \myterminalScript{ros2 run first_package_py publisher_py} because the Node was run from \myfileScript{setup.py}. We do not have \myfileScript{setup.py} in this package, so we need to run the script directly with  \myterminalScript{ros2 run first_package publisher_py.py}
    \end{compactitem}
\vspace*{\fill}
\end{frame}


%####################################
\section{Launch Files}
\begin{frame}[fragile]{\sec}
\vspace*{\fill}
{\footnotesize \bf \mydefinition \raleway Launch Files}\doublerulefill
\begin{mdframed}
\raleway \scriptsize
\urllink{https://docs.ros.org/en/foxy/Tutorials/Intermediate/Launch/Creating-Launch-Files.html}{Launch files} in ROS2 can be written in XML, YAML, or Python. With Python it is easier to add logic (\pscript{if}, \pscript{else}, \pscript{for},  \pscript{while}, etc), functions, events, \pscript{print}, etc. Here is an \urllink{https://github.com/ros2/launch_ros/blob/foxy/launch_ros/examples/lifecycle_pub_sub_launch.py}{example} of a complex launch file written in Python.
\end{mdframed}

{\footnotesize \bf \mytodo \raleway Create a Launch File}\doublerulefill
\begin{mdframed}
\raleway \scriptsize
\begin{compactitem}
    \scriptsize
\item In the package \mypackageScript{first_package}, create a \mydirScript{launch} folder.    
\item In this folder, create the file \myfileScript{first.launch.py}
\end{compactitem}
\end{mdframed}

{\footnotesize \bf \mytodo \raleway Edit the Launch File}\doublerulefill
\begin{mdframed}
\raleway \scriptsize
\begin{compactitem}
    \scriptsize
\item Edit \myfileScript{first.launch.py} to start the publisher and the subscriber.
\end{compactitem}

\end{mdframed}
\vspace*{\fill}
\end{frame}


%####################################

\begin{frame}[fragile]{\sec}
\vspace*{\fill}
{\footnotesize \bf \mydefinition \raleway Minimal Example \#1}\doublerulefill
\begin{mdframed}
\raleway \scriptsize
Example of a minimal working launch file to run one publisher and one subscriber.
\end{mdframed}

\begin{pyscriptII}
# pull in some Python launch modules.
from launch import LaunchDescription  
from launch_ros.actions import Node

# this function is needed
def generate_launch_description(): 
    ld = LaunchDescription()            # instantiate a Launchdescription object
    publisher_node = Node(              # declare your Node
        package="first_package",        # package name
        executable="publisher_py.py"    # executable as set in setup.py
    )
    subscriber_node = Node(
        package="first_package",
        executable="subscriber_cpp"
    )
    ld.add_action(publisher_node)  # add each Node to the LaunchDescription object
    ld.add_action(subscriber_node) # add each Node to the LaunchDescription object
    return ld                      # return the LaunchDescription object
\end{pyscriptII}

{\footnotesize \bf \mywarning \raleway Compile Package}\doublerulefill
\begin{mdframed}
\raleway \scriptsize
Always rebuild your package after modifying the launch file.
\end{mdframed}
\vspace*{\fill}
\end{frame}

\begin{frame}[fragile]{\sec}
\vspace*{\fill}
{\footnotesize \bf \mydefinition \raleway Minimal Example \#2}\doublerulefill
\begin{mdframed}
\raleway \scriptsize
This example is similar to the example from the previous slide. Instead of creating a \pscript{LaunchDescription()} object and returnint it, we return the call to \pscript{LaunchDescription()} directly.
\end{mdframed}

\begin{pyscriptII}
from launch import LaunchDescription
from launch_ros.actions import Node

def generate_launch_description():
    return LaunchDescription([
        Node(
            package="first_package",
            executable="publisher_py.py",
        ),
        Node(
            package="first_package",
            executable="subscriber_cpp",
        ),
    ])
\end{pyscriptII}


\vspace*{\fill}
\end{frame}


\begin{frame}[fragile]{\sec}
\vspace*{\fill}
{\footnotesize \bf \mytodo \raleway Edit CMakeLists.txt}\doublerulefill
\begin{mdframed}
\raleway \scriptsize
Since we created a new folder \mydirScript{launch}, we need to install it as well. Edit \myfileScript{CMakeLists.txt} from \mypackageScript{first_package}
\end{mdframed}

\begin{cmakescript}
install(DIRECTORY include launch DESTINATION share/${PROJECT_NAME})
\end{cmakescript}

{\footnotesize \bf \mytodo \raleway Run Launch File}\doublerulefill
\begin{mdframed}
\raleway \scriptsize
\myterminalScript{colcon build -{}-packages-select first_package}

\myterminalScript{ros2 launch first_package first.launch.py}
\end{mdframed}

\vspace*{\fill}
\end{frame}


%%%%%%%%%%%%%%%%%%
%   --- FRAME
%%%%%%%%%%%%%%%%%%
\section{Next Class}
\begin{frame}[fragile]{\sec}
\vspace*{\fill}
\begin{center} 
\Large

\myinfo~\cnordZero{L4:} ROS -- Part III.

\begin{itemize}
\normalsize
    \item More on Launch Files.
    \item Service Clients.
    \item Custom Messages.
    \item Multithreaded Executors.
    \item Parameters.
\end{itemize}

\end{center}
\vspace*{\fill}
\end{frame}
%=+=+=+=+=+=+=+=+=+=+=+=+=+=+=+=+=+=+=+=+=+=+=+=+=+=+=+=+=+=+=+=+=+=+=+=+=+=+=+=
% END OF DOCUMENT
%=+=+=+=+=+=+=+=+=+=+=+=+=+=+=+=+=+=+=+=+=+=+=+=+=+=+=+=+=+=+=+=+=+=+=+=+=+=+=+=
\end{document}